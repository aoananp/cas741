\documentclass[12pt, titlepage]{article}

\usepackage[margin=1.0in]{geometry}
\usepackage{amsmath, mathtools}
\usepackage{amsfonts}
\usepackage{amssymb}
\usepackage{graphicx}
\usepackage{colortbl}
\usepackage{xr}
\usepackage{hyperref}
\usepackage{longtable}
\usepackage{xfrac}
\usepackage{tabularx}
\usepackage{float}
\usepackage{siunitx}
\usepackage{pdflscape}
\usepackage{afterpage}
\usepackage{booktabs}
\usepackage{caption}
\hypersetup{
    colorlinks,
    citecolor=black,
    filecolor=black,
    linkcolor=red,
    urlcolor=blue
}
\usepackage[round]{natbib}

% For easy change of table widths
\newcommand{\colZwidth}{1.0\textwidth}
\newcommand{\colAwidth}{0.13\textwidth}
\newcommand{\colBwidth}{0.82\textwidth}
\newcommand{\colCwidth}{0.1\textwidth}
\newcommand{\colDwidth}{0.05\textwidth}
\newcommand{\colEwidth}{0.8\textwidth}
\newcommand{\colFwidth}{0.17\textwidth}
\newcommand{\colGwidth}{0.5\textwidth}
\newcommand{\colHwidth}{0.28\textwidth}

\newcounter{tnum} %Functional Test
\newcommand{\tthetestnum}{\thetnum}
\newcommand{\tref}[1]{T\ref{#1}}

%% Comments

\usepackage{color}

\newif\ifcomments\commentstrue

\ifcomments
\newcommand{\authornote}[3]{\textcolor{#1}{[#3 ---#2]}}
\newcommand{\todo}[1]{\textcolor{red}{[TODO: #1]}}
\else
\newcommand{\authornote}[3]{}
\newcommand{\todo}[1]{}
\fi

\newcommand{\wss}[1]{\authornote{blue}{SS}{#1}}
\newcommand{\wpa}[1]{\authornote{magenta}{PA}{#1}}

\newcommand{\famname}{LODES} % PUT YOUR PROGRAM NAME HERE
\newcommand{\famdesc}{Library of ODE Solvers}
\newcommand{\famurl}{https://github.com/aoananp/cas741/}

\begin{document}

\title{Test Plan for the Library of ODE Solvers (LODES)} 
\author{Paul Aoanan}
\date{\today}
	
\maketitle

\pagenumbering{roman}

\section{Revision History}
\begin{tabularx}{\textwidth}{p{3cm}p{2cm}X}
\toprule {\bf Date} & {\bf Version} & {\bf Notes}\\
\midrule
\today & 1.0 & Initial draft.\\
%Date 2 & 1.1 & Notes\\
\bottomrule
\end{tabularx}

~\newpage

\section{Symbols, Abbreviations and Acronyms}
The following table lists the symbols, abbreviations and acronyms used in the Test Plan.
The software library's Commonality Analysis (CA) tables provide supplementary items in addition to the ones listed below.\\
\begin{table} [h]
\renewcommand{\arraystretch}{1.2}
\begin{tabular}{l l l l |} 
  \toprule		
  \textbf{symbol} & \textbf{description}\\
  \midrule 
  CA & Commonality Analysis\\
  IDE & Integrated Development Environment\\ 
  IVP & Initial Value Problem\\
  ODE & Ordinary Differential Equation\\
  \famname{} & \famdesc{}\\
  SRS & Software Requirements Specification\\
  T & Test\\
  O & Output\\
  \bottomrule
\end{tabular}\\
  \caption{Symbols, Abbreviations, and Acronyms used in the Test Plan}
  \label{Table:Table_Symbols}
\end{table}

~\newpage

\tableofcontents

\listoftables

\listoffigures

\newpage

\pagenumbering{arabic}

%This document ...

\section{General Information}
The following section provides an overview of the Test Plan for the Library of Ordinary Differential Equation (ODE) Solvers.\\
This section explains the purpose of this document, the scope of the system, and an overview of the following sections.

\subsection{Purpose}
The main purpose of this document (the Test Plan) is to describe the verification and validation process that will be used to test the
functionality of \famname{}.  This document closely follows the requirements and governs the subsequent testing activities.
This document is intended to be used as a reference for all testing and will be used to increase confidence in the software implementation.\\
\\
This document will be used as a guide and starting point for the Test Report. The test cases
listed in this document will be executed and the output will be analyzed to uncover errors, increase confidence and correctness in the software.

\subsection{Scope}
The scope of the testing is limited to the \famdesc{}. Given the appropriate inputs, each program in \famname{} is intended to find
the solution to an Initial Value Problem (IVP).\\
\\
Due to time and cost constraints, the scope of testing is limited to automated unit and manual system verification and validation activities.\\
\\
Static testing will be briefly described and will be left to the developer and verifier to perform with due diligence.

\subsection{Overview of Document}
The following sections provide more detail about the testing of \famname{}.
Information about the testing process is provided and the software specifications
that were discussed in the Commonality Analysis are stated.
The evaluation process that will be followed during testing is outlined and test cases
for both the system testing and unit testing are provided.

\section{Plan}
This section provides a description of the software that is being tested, the team that will
perform the testing, the approach to automated testing, the tools to be used for verification,
and the non-testing based verification. 
	
\subsection{Software Description}
The software being tested is the \famdesc{}. Given the ODE, initial values of $x$ and $y$, and the final value of $x$,
the programs calculate the final value of $y$ through the use of numerical methods.

\subsection{Test Team}

The test team that will execute the test cases, write and review the Test Report consists of:
\begin{itemize}
 \item Paul Aoanan
 \item To be determined (The test report will be reviewed by an independent individual)
\end{itemize} 

\subsection{Automated Testing Approach}
Automated unit testing will be implemented for \famname{} as described in Section \ref{sec_verificationtools}.
%Since MATLAB only offers commercially available testing libraries, the scope of automated testing in this Test
%Plan is limited to unit testing.

\subsection{Verification Tools} \label{sec_verificationtools}
The verification tools to be used will be the following:

\begin{enumerate}
\item{Unit Testing Framework\\}
A Unit Testing Framework designed in MATLAB that will compare MATLAB's own functional programs
with
\famname{}' running the same inputs will be implemented.

The following algorithm will be implemented to compare the results:
$$\epsilon_{\text{relative}} = \frac{\text{Result}_\text{MATLAB} - \text{Result}_\text{\famname{}}} {\text{Result}_\text{MATLAB}} $$

\item{Static Analyer\\}
The program's IDE (MATLAB) will be used as a Static Analyzer tool for program debugging and
for checking syntax errors.

\item{Continuous Integration\\}
The source code and the project repository is located in GitHub at: \url{\famurl}.
It provides the Build Server functionality to fully maintain and document the software through its lifecycle.
As well, it provides the compare functionality for future regression testing and analysis of code updates.

\item{Code Coverage Tool\\}
Due to the commercial nature of MATLAB, only commercial code coverage tools are viable for use due to the maturity, increased confidence, and detailed documentation that they offer. Other coverage tools may be considered, but no code coverage tool will be considered in the scope of this test plan due to budget constraints.

\end{enumerate}

%\wss{Thoughts on what tools to use, such as the following: unit testing
%  framework, valgrind, static analyzer, make, continuous integration, test
%  coverage tool, etc.}

% \subsection{Testing Schedule}
		
% See Gantt Chart at the following url ...

\subsection{Non-Testing Based Verification}
\famname{} will undergo the following non-testing based verification activities:

\paragraph{Code Inspection\\}
\famname{} will undergo an initial desk review by an independent body.
The code will be perused for syntax errors and correct program calls.
This code inspection activity provides the initial sanity check for the developer and the software.

\paragraph{Code Walkthrough\\}
\famname{} will undergo code walkthrough by the developer and an independent body.
They will jointly review the code and reference the Commonality Analysis for algorithm adherence. This activity
will also involve logic analysis, loop and recursion boundary tracing (using by-hand test cases), passing of
variables and references, and if the code is programmatically correct.

\paragraph{Symbolic Execution\\}
Generally, symbolic execution will be performed using the boundary input conditions. Test conditions will be
analyzed and executed by-hand. Generally, inputs in, on, and around the boundary conditions are chosen.

%\wss{List any approaches like code inspection, code walkthrough, symbolic
%execution etc.  Enter not applicable if that is the case.}

\section{System Test Description}
System testing will be executed to provide increased confidence that \famname{} will achieve the goals defined in the Commonality Analysis. It uses a "black box" approach wherein it tests the system as a whole through the use of input and output analysis.

\subsection{Tests for Faulty Input}

\subsubsection{Input}
		
The input will be based on the Assumptions table in the Commonality Analysis. Each test will correspond to an entry from the assumptions item whilst altering a specific input variable to a non-permissible value. The list of inputs is in order with the entries in the table.

\begin{table} [H]
  \caption{Faulty Input Test Cases}
  \label{Table:Table_FaultyInputs}  
\begin{tabular}{|c|p{8cm}|p{6cm}|}
  \hline	
  \textbf{Number} & \textbf{Input} &\textbf{Expected Outcome}\\
  \hline 
  01 & ODE Function Call $\mathbb{\notin}$ \{euler741, trap741, heun741, rk4741\} $\mathbb{\cup}$ 
  MATLAB functions & error: undefined function call\\ \hline
  02& $f(x, y) = y'' + y' + x + 2$ & success: false\\ \hline
  03& $f(x, y) = y' + 1$ & success: false\\ \hline
  04& $f(x, y) = (dx/dy) + 1$ & success: false\\ \hline
  05& $f(x, y) = (y + 1) / x$ & success: false\\ \hline
  06& $f(x, y) = y/(x-3)$ & success: false\\ \hline
  07& Boundary Value Problem & success: false\\ \hline
  08& $h = -1$ & success: false\\ \hline
  09& $h = 0$ & success: false\\ \hline
  10& $x_0 = i$ & success: false\\ \hline 
  11& $x_0 = [0, 1]$ & success: false\\ \hline 
  12& $y_0 = i$ & success: false\\ \hline 
  13& $y_0 = [0, 1]$ & success: false\\ \hline
  14& $x_k = i$ & success: false\\ \hline 
  15& $x_k = [0, 1]$ & success: false\\ \hline

\end{tabular}\\
\end{table}

\subsection{Tests for Functional Requirements}

\subsubsection{Calculation Tests}
		
\paragraph{Euler's Method}

\begin{enumerate}

\item{\textbf{T-\refstepcounter{tnum}\thetnum \label{t-euler_simple}: Simple Case}}

Type: Functional, Automated, System %Functional, Dynamic, Manual, Static etc.
					
Initial State: Not applicable
					
Input: $f(x, y) = y, h = 2, x_0 = 0, y_0 = 1, x_k = 2$
					
Output: $y_k = 3$, success = true
					
How test will be performed: Automated system test

\item{\textbf{T-\refstepcounter{tnum}\thetnum \label{t-euler_simpleiterative}: Simple-Iterative Case}}

Type: Functional, Automated, System %Functional, Dynamic, Manual, Static etc.
					
Initial State: Not applicable
					
Input: $f(x, y) = y, h = 0.5, x_0 = 0, y_0 = 1, x_k = 2$
					
Output: $y_k = 5.0625$, success = true
					
How test will be performed: Automated system test

\item{\textbf{T-\refstepcounter{tnum}\thetnum \label{t-euler_nonlinear}: Non-linear Case}}

Type: Functional, Automated, System %Functional, Dynamic, Manual, Static etc.
					
Initial State: Not applicable
					
Input: $f(x, y) = sin(x) - y^2, h = 5, x_0 = 0, y_0 = 1, x_k = 5$
					
Output: $y_k = -4$, success = true
					
How test will be performed: Automated system test

\item{\textbf{T-\refstepcounter{tnum}\thetnum \label{t-euler_nonlineariterative}: Non-linear Iterative Case}}

Type: Functional, Automated, System %Functional, Dynamic, Manual, Static etc.
					
Initial State: Not applicable
					
Input: $f(x, y) = sin(x) - y^2, h = 1, x_0 = 0, y_0 = 1, x_k = 5$
					
Output: $y_k = -0.6695$, success = true
					
How test will be performed: Automated system test

\end{enumerate}

\paragraph{Trapezoid Method}
\begin{enumerate}

\item{\textbf{T-\refstepcounter{tnum}\thetnum \label{t-trap_simple}: Simple Case}}

Type: Functional, Automated, System %Functional, Dynamic, Manual, Static etc.
					
Initial State: Not applicable
					
Input: $f(x, y) = y, h = 2, x_0 = 0, y_0 = 1, x_k = 2$
					
Output: $y_k = 3$, success = true
					
How test will be performed: Automated system test

\item{\textbf{T-\refstepcounter{tnum}\thetnum \label{t-trap_simpleiterative}: Simple-Iterative Case}}

Type: Functional, Automated, System %Functional, Dynamic, Manual, Static etc.
					
Initial State: Not applicable
					
Input: $f(x, y) = y, h = 0.5, x_0 = 0, y_0 = 1, x_k = 2$
					
Output: $y_k = 5.0625$, success = true
					
How test will be performed: Automated system test

\item{\textbf{T-\refstepcounter{tnum}\thetnum \label{t-trap_nonlinear}: Non-linear Case}}

Type: Functional, Automated, System %Functional, Dynamic, Manual, Static etc.
					
Initial State: Not applicable
					
Input: $f(x, y) = sin(x) - y^2, h = 5, x_0 = 0, y_0 = 1, x_k = 5$
					
Output: $y_k = -4$, success = true
					
How test will be performed: Automated system test

\item{\textbf{T-\refstepcounter{tnum}\thetnum \label{t-trap_nonlineariterative}: Non-linear Iterative Case}}

Type: Functional, Automated, System %Functional, Dynamic, Manual, Static etc.
					
Initial State: Not applicable
					
Input: $f(x, y) = sin(x) - y^2, h = 1, x_0 = 0, y_0 = 1, x_k = 5$
					
Output: $y_k = -0.6695$, success = true
					
How test will be performed: Automated system test

\end{enumerate}

\paragraph{Heun's Method}
\begin{enumerate}

\item{\textbf{T-\refstepcounter{tnum}\thetnum \label{t-heun_simple}: Simple Case}}

Type: Functional, Automated, System %Functional, Dynamic, Manual, Static etc.
					
Initial State: Not applicable
					
Input: $f(x, y) = y, h = 2, x_0 = 0, y_0 = 1, x_k = 2$
					
Output: $y_k = 3$, success = true
					
How test will be performed: Automated system test

\item{\textbf{T-\refstepcounter{tnum}\thetnum \label{t-heun_simpleiterative}: Simple-Iterative Case}}

Type: Functional, Automated, System %Functional, Dynamic, Manual, Static etc.
					
Initial State: Not applicable
					
Input: $f(x, y) = y, h = 0.5, x_0 = 0, y_0 = 1, x_k = 2$
					
Output: $y_k = 5.0625$, success = true
					
How test will be performed: Automated system test

\item{\textbf{T-\refstepcounter{tnum}\thetnum \label{t-heun_nonlinear}: Non-linear Case}}

Type: Functional, Automated, System %Functional, Dynamic, Manual, Static etc.
					
Initial State: Not applicable
					
Input: $f(x, y) = sin(x) - y^2, h = 5, x_0 = 0, y_0 = 1, x_k = 5$
					
Output: $y_k = -4$, success = true
					
How test will be performed: Automated system test

\item{\textbf{T-\refstepcounter{tnum}\thetnum \label{t-heun_nonlineariterative}: Non-linear Iterative Case}}

Type: Functional, Automated, System %Functional, Dynamic, Manual, Static etc.
					
Initial State: Not applicable
					
Input: $f(x, y) = sin(x) - y^2, h = 1, x_0 = 0, y_0 = 1, x_k = 5$
					
Output: $y_k = -0.6695$, success = true
					
How test will be performed: Automated system test

\end{enumerate}

\paragraph{Runge-Kutta Method}
\begin{enumerate}

\item{\textbf{T-\refstepcounter{tnum}\thetnum \label{t-rk_simple}: Simple Case}}

Type: Functional, Automated, System %Functional, Dynamic, Manual, Static etc.
					
Initial State: Not applicable
					
Input: $f(x, y) = y, h = 2, x_0 = 0, y_0 = 1, x_k = 2$
					
Output: $y_k = 3$, success = true
					
How test will be performed: Automated system test

\item{\textbf{T-\refstepcounter{tnum}\thetnum \label{t-rk_simpleiterative}: Simple-Iterative Case}}

Type: Functional, Automated, System %Functional, Dynamic, Manual, Static etc.
					
Initial State: Not applicable
					
Input: $f(x, y) = y, h = 0.5, x_0 = 0, y_0 = 1, x_k = 2$
					
Output: $y_k = 5.0625$, success = true
					
How test will be performed: Automated system test

\item{\textbf{T-\refstepcounter{tnum}\thetnum \label{t-rk_nonlinear}: Non-linear Case}}

Type: Functional, Automated, System %Functional, Dynamic, Manual, Static etc.
					
Initial State: Not applicable
					
Input: $f(x, y) = sin(x) - y^2, h = 5, x_0 = 0, y_0 = 1, x_k = 5$
					
Output: $y_k = -4$, success = true
					
How test will be performed: Automated system test

\item{\textbf{T-\refstepcounter{tnum}\thetnum \label{t-rk_nonlineariterative}: Non-linear Iterative Case}}

Type: Functional, Automated, System %Functional, Dynamic, Manual, Static etc.
					
Initial State: Not applicable
					
Input: $f(x, y) = sin(x) - y^2, h = 1, x_0 = 0, y_0 = 1, x_k = 5$
					
Output: $y_k = -0.6695$, success = true
					
How test will be performed: Automated system test

\end{enumerate}

%\subsubsection{Area of Testing2}
%
%...

\subsection{Tests for Nonfunctional Requirements}

\subsubsection{Performance Requirements}
		
\paragraph{Speed}

\begin{enumerate}

\item{\textbf{T-\refstepcounter{tnum}\thetnum \label{t-speed}: Speed Benchmark}}

Type: Non-Functional, Automated, Performance 
					
Initial State: Not Applicable
					
Input/Condition: $f(x, y) = sin(x) - y^2, h = 1E-5, x_0 = 0, y_0 = 1, x_k = 5$
					
Output/Result: $\sigma_\text{\famname{}} \leq 4*\sigma_\text{MATLAB}$ (\famname{}' runtimes shall be no more than four (4) times that of MATLAB's.)
					
How test will be performed: Using MATLAB's Run and Time functionality, the execution time of a program will be measured and compared through program calls to the respective MATLAB and \famname{} functions.\

\end{enumerate}

\subsubsection{Results Analysis}

\paragraph{Benchmark Results}

\begin{enumerate}

\item{\textbf{T-\refstepcounter{tnum}\thetnum \label{t-benchmark}: MATLAB Benchmark}}

Type: Non-Functional, Automated, Precision 
					
Initial State: Not Applicable
					
Input/Condition: $f(x, y) = sin(x) - y^2, h = \text{(VARIABLE)}, x_0 = 0, y_0 = 1, x_k = 5$
					
Output/Result: The $\epsilon_{\text{relative}}$ vs. $h$ plot.
The $y_k$ values will be compared according to the following formula -
$$\epsilon_{\text{relative}} = \frac{\text{Result}_\text{MATLAB} - \text{Result}_\text{\famname{}}} {\text{Result}_\text{MATLAB}} $$
					
How test will be performed: The $h$ (step-size) values will be varied across the range (0, 1000]. 
$\epsilon_{\text{relative}}$ will be plotted against $h$.

\end{enumerate}

%\item{test-id2\\}
%
%Type: Functional, Dynamic, Manual, Static etc.
%					
%Initial State: 
%					
%Input: 
%					
%Output: 
%					
%How test will be performed: 

%\subsubsection{Area of Testing2}
%
%...

\subsection{Traceability Between Test Cases and Requirements}
The following table shows the traceability mapping for the test cases laid out in this Test Plan to the requirements described in the Commonality Analysis.

\begin{table} [H]
  \caption{Requirements Traceability Matrix}
  \label{Table:Table_Traceability}  
\begin{tabular}{|c|p{12cm}|}
  \hline	
  \textbf{Test Number} & \textbf{CA Requirement}\\
  \hline 
   T1& ODE F\\ \hline
   T2& ODE F\\ \hline
   T3& ODE F\\ \hline
   T4& ODE F\\ \hline
   T5& ODE F\\ \hline
   T6& ODE F\\ \hline
   T7& ODE F\\ \hline
   T8& ODE F\\ \hline
   T9& ODE F\\ \hline
   T10& ODE F\\ \hline
   T11& ODE F\\ \hline
   T12& ODE F\\ \hline
   T13& ODE F\\ \hline
   T14& ODE F\\ \hline
   T15& ODE F\\ \hline
   T16& ODE F\\ \hline
   T17& ODE F\\ \hline
   T18& ODE F\\ \hline

\end{tabular}\\
\end{table}
				
\section{Unit Testing Plan}
		
%\wss{Unit testing plans for internal functions and, if appropriate, output
% files}

\subsubsection{ODE String Parser}

\paragraph{Parser Functionality}

\begin{enumerate}

\item{\textbf{T-\refstepcounter{tnum}\thetnum \label{t-parser1}: Simple}}

Type: Functional, Manual, Unit 
					
Initial State: Not Applicable
					
Input/Condition: $f(x, y) = x^2 + y^2 + 1$
					
Output/Result: Machine-interpreted $f(x, y)$
					
How test will be performed: The parser will be prov

\end{enumerate}
		
\paragraph{Error Analysis Through Comparison}

\begin{enumerate}

\item{\textbf{T-\refstepcounter{tnum}\thetnum \label{t-difference}: Error Analysis}}

Type: Manual, Comparison 
					
Initial State: Not Applicable
					
Input/Condition: $f(x, y) = y^2 - x, h = 1E-7, x_0 = 0, y_0 = 1, x_k = 1$
					
Output/Result: Plot of $\Delta_{\text{relative}}$ vs. $h$

How test will be performed: Using the matlab.perftest.TimeExperiment() function, the execution time of a program will be measured and compared through program calls to the respective MATLAB and \famname{} functions.

\end{enumerate}

\bibliographystyle{plainnat}

\bibliography{SRS}

\newpage

\section{Appendix}

This is where you can place additional information.

\subsection{Symbolic Parameters}

The definition of the test cases will call for SYMBOLIC\_CONSTANTS.
Their values are defined in this section for easy maintenance.

\subsection{Usability Survey Questions?}

This is a section that would be appropriate for some teams.

\end{document}
