\documentclass[12pt, titlepage]{article}

\usepackage[margin=1.0in]{geometry}
\usepackage{amsmath, mathtools}
\usepackage{amsfonts}
\usepackage{amssymb}
\usepackage{graphicx}
\usepackage{colortbl}
\usepackage{xr}
\usepackage{hyperref}
\usepackage{longtable}
\usepackage{xfrac}
\usepackage{tabularx}
\usepackage{float}
\usepackage{siunitx}
\usepackage{pdflscape}
\usepackage{afterpage}
\usepackage{booktabs}
\usepackage{caption}
\hypersetup{
    colorlinks,
    citecolor=black,
    filecolor=black,
    linkcolor=red,
    urlcolor=blue
}
\usepackage[round]{natbib}

% For easy change of table widths
\newcommand{\colZwidth}{1.0\textwidth}
\newcommand{\colAwidth}{0.13\textwidth}
\newcommand{\colBwidth}{0.82\textwidth}
\newcommand{\colCwidth}{0.1\textwidth}
\newcommand{\colDwidth}{0.05\textwidth}
\newcommand{\colEwidth}{0.8\textwidth}
\newcommand{\colFwidth}{0.17\textwidth}
\newcommand{\colGwidth}{0.5\textwidth}
\newcommand{\colHwidth}{0.28\textwidth}

\newcounter{ftnum} %Functional Test
\newcommand{\ftthefttestnum}{\theftnum}
\newcommand{\ftref}[1]{FT\ref{#1}}

\input{../Comments}
\newcommand{\famname}{LODES} % PUT YOUR PROGRAM NAME HERE
\newcommand{\famdesc}{Library of ODE Solvers}
\newcommand{\famurl}{https://github.com/aoananp/cas741/}

\begin{document}

\title{Test Plan for the Library of ODE Solvers (LODES)} 
\author{Paul Aoanan}
\date{\today}
	
\maketitle

\pagenumbering{roman}

\section{Revision History}
\begin{tabularx}{\textwidth}{p{3cm}p{2cm}X}
\toprule {\bf Date} & {\bf Version} & {\bf Notes}\\
\midrule
\today & 1.0 & Initial draft.\\
%Date 2 & 1.1 & Notes\\
\bottomrule
\end{tabularx}

~\newpage

\section{Symbols, Abbreviations and Acronyms}
The following table lists the symbols, abbreviations and acronyms used in the Test Plan.
The software library's Commonality Analysis (CA) tables provide supplementary items in addition to the ones listed below.\\
\begin{table} [h]
\renewcommand{\arraystretch}{1.2}
\begin{tabular}{l l l l |} 
  \toprule		
  \textbf{symbol} & \textbf{description}\\
  \midrule 
  CA & Commonality Analysis\\
  IDE & Integrated Development Environment\\ 
  IVP & Initial Value Problem\\
  ODE & Ordinary Differential Equation\\
  \famname{} & \famdesc{}\\
  SRS & Software Requirements Specification\\
  T & Test\\
  O & Output\\
  \bottomrule
\end{tabular}\\
  \caption{Symbols, Abbreviations, and Acronyms used in the Test Plan}
  \label{Table:Table_Symbols}
\end{table}

~\newpage

\tableofcontents

\listoftables

\listoffigures

\newpage

\pagenumbering{arabic}

%This document ...

\section{General Information}
The following section provides an overview of the Test Plan for the Library of Ordinary Differential Equation (ODE) Solvers.\\
This section explains the purpose of this document, the scope of the system, and an overview of the following sections.

\subsection{Purpose}
The main purpose of this document (the Test Plan) is to describe the verification and validation process that will be used to test the
functionality of \famname{}.  This document closely follows the requirements and governs the subsequent testing activities.
This document is intended to be used as a reference for all testing and will be used to increase confidence in the software implementation.\\
\\
This document will be used as a guide and starting point for the Test Report. The test cases
listed in this document will be executed and the output will be analyzed to uncover errors, increase confidence and correctness in the software.

\subsection{Scope}
The scope of the testing is limited to the \famdesc{}. Given the appropriate inputs, each program in \famname{} is intended to find
the solution to an Initial Value Problem (IVP).\\

\subsection{Overview of Document}
The following sections provide more detail about the testing of \famname{}.
Information about the testing process is provided and the software specifications
that were discussed in the Commonality Analysis are stated.
The evaluation process that will be followed during testing is outlined and test cases
for both the system testing and unit testing are provided.

\section{Plan}
This section provides a description of the software that is being tested, the team that will
perform the testing, the approach to automated testing, the tools to be used for verification, and the non-testing based verification. 
	
\subsection{Software Description}
The software being tested is the \famdesc{}. Given the ODE, initial values of $x$ and $y$, and the final value of $x$,
the programs calculate the final value of $y$ through the use of numerical methods.

\subsection{Test Team}

The test team that will execute the test cases, write and review the Test Report consists of:
\begin{itemize}
 \item Paul Aoanan
\end{itemize} 

\subsection{Automated Testing Approach}



\subsection{Verification Tools}
The verification tools to be used will be the following:

\begin{enumerate}
\item{Unit Testing Framework\\}
A Unit Testing Framework designed in MATLAB that will compare MATLAB's own functional programs
with
\famname{}' running the same inputs will be implemented.

\item{Static Analyer\\}
The program's IDE (MATLAB) will be used as a Static Analyzer tool for program debugging and
for checking syntax errors.

\item{Continuous Integration\\}
The source code and the project repository is located in GitHub at: \url{\famurl}.
It provides the Build Server functionality to fully maintain and document the software through its lifecycle.
As well, it provides the compare functionality for future regression testing and analysis of code updates.

\end{enumerate}

\wss{Thoughts on what tools to use, such as the following: unit testing
  framework, valgrind, static analyzer, make, continuous integration, test
  coverage tool, etc.}

% \subsection{Testing Schedule}
		
% See Gantt Chart at the following url ...

\subsection{Non-Testing Based Verification}
The \famname{} will undergo 

\paragraph{Code Inspection}
The \famname{} will initiallyh 

\wss{List any approaches like code inspection, code walkthrough, symbolic
  execution etc.  Enter not applicable if that is the case.}

\section{System Test Description}
	
\subsection{Tests for Functional Requirements}

\subsubsection{Initialization}
		
\paragraph{Paragraph}

\begin{enumerate}
%
%\item{FT-\refstepcounter{ftnum}\theftnum \label{ft-syntax}}
%
%Type: Static %Functional, Dynamic, Manual, Static etc.
%					
%Initial State: The source code "as-is"
%					
%Input: The source code as the input to this test
%					
%Output: The code's adherence to the language syntax rules
%					
%How test will be performed: The code will undergo desk review to check for missing and incorrect syntax
%
%\item{FT-\refstepcounter{ftnum}\theftnum \label{ft-walkthrough}}
%
%Type: Static %Functional, Dynamic, Manual, Static etc.
%					
%Initial State: The source code "as-is"
%					
%Input: The source code as the input to this test
%					
%Output: The code's adherence to the goals of the program.
%					
%How test will be performed: The code will undergo desk review to determine, in closed bounds and within reason, if the code satisfies the machine implementation of the mathematical formulas listed in the Commonality Analysis.
%
%\item{FT-\refstepcounter{ftnum}\theftnum \label{ft-walkthrough}}
%
%Type: Static %Functional, Dynamic, Manual, Static etc.
%					
%Initial State: The source code "as-is"
%					
%Input: The source code as the input to this test
%					
%Output: The code's adherence to the goals of the program.
%					
%How test will be performed: The code will undergo desk review to determine, in closed bounds and within reason, if the code satisfies the machine implementation of the mathematical formulas listed in the Commonality Analysis.
					
\item{FT-\refstepcounter{ftnum}\theftnum \label{ft-}}

Type: Functional, Dynamic, Manual, Static etc. %Functional, Dynamic, Manual, Static etc.
					
Initial State: 
					
Input: 
					
Output: 
					
How test will be performed: 

\end{enumerate}

\subsubsection{Area of Testing2}

...

\subsection{Tests for Nonfunctional Requirements}

\subsubsection{Area of Testing1}
		
\paragraph{Title for Test}

\begin{enumerate}

\item{test-id1\\}

Type: 
					
Initial State: 
					
Input/Condition: 
					
Output/Result: 
					
How test will be performed: 
					
\item{test-id2\\}

Type: Functional, Dynamic, Manual, Static etc.
					
Initial State: 
					
Input: 
					
Output: 
					
How test will be performed: 

\end{enumerate}

\subsubsection{Area of Testing2}

...

\subsection{Traceability Between Test Cases and Requirements}

% \section{Tests for Proof of Concept}

% \subsection{Area of Testing1}
		
% \paragraph{Title for Test}

% \begin{enumerate}

% \item{test-id1\\}

% Type: Functional, Dynamic, Manual, Static etc.
					
% Initial State: 
					
% Input: 
					
% Output: 
					
% How test will be performed: 
					
% \item{test-id2\\}

% Type: Functional, Dynamic, Manual, Static etc.
					
% Initial State: 
					
% Input: 
					
% Output: 
					
% How test will be performed: 

% \end{enumerate}

% \subsection{Area of Testing2}

% ...
				
\section{Unit Testing Plan}
		
\wss{Unit testing plans for internal functions and, if appropriate, output
  files}

\bibliographystyle{plainnat}

\bibliography{SRS}

\newpage

\section{Appendix}

This is where you can place additional information.

\subsection{Symbolic Parameters}

The definition of the test cases will call for SYMBOLIC\_CONSTANTS.
Their values are defined in this section for easy maintenance.

\subsection{Usability Survey Questions?}

This is a section that would be appropriate for some teams.

\end{document}
