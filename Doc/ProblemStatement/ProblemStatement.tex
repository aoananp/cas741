\documentclass{article}

\usepackage{tabularx}
\usepackage{booktabs}

\title{CAS 741: Problem Statement\\Analysis and Comparison of a Family of Single ODE Solvers}

\author{Paul Aoanan - 000952218}

\date{September 15, 2017}

%% Comments

\usepackage{color}

\newif\ifcomments\commentstrue

\ifcomments
\newcommand{\authornote}[3]{\textcolor{#1}{[#3 ---#2]}}
\newcommand{\todo}[1]{\textcolor{red}{[TODO: #1]}}
\else
\newcommand{\authornote}[3]{}
\newcommand{\todo}[1]{}
\fi

\newcommand{\wss}[1]{\authornote{blue}{SS}{#1}}
\newcommand{\wpa}[1]{\authornote{magenta}{PA}{#1}}


\begin{document}

\maketitle

\begin{table}[hp]
\caption{Revision History} \label{TblRevisionHistory}
\begin{tabularx}{\textwidth}{llX}
\toprule
\textbf{Date} & \textbf{Developer(s)} & \textbf{Change}\\
\midrule
September 15, 2017 & Paul Aoanan & Initial Draft\\
%Date2 & Name(s) & Description of changes\\
%... & ... & ...\\
\bottomrule
\end{tabularx}
\end{table}

In physical sciences, mathematical models are derived from scientific models to represent a real world phenomenon through formal mathematical constructs.

Scientific models in the study of radioactivity, carbon decay, and Newton's Law of Cooling
involve the use of ordinary differential equations (ODEs).

Known elementary techniques of solving ODEs use the linear approximation method wherein the
solution is based upon assuming or ``approximating" the slope of the tangent line from one
reference point to the next until the target point has been reached.

Since these methods only yield approximate solutions, these methods introduce an error term to
the solution.

One of the sources of error in solving ODEs is the arbitrary step-size.
The goal is not to minimize the time-step, but to solve the equation using a step-size that will yield an acceptable error.

Multiple methods in solving ODEs exist.
Examples of which are the following:
\begin{itemize}
	\item Euler's Method
	\item Heun's Method (Modified Euler's Method)
	\item Fourth Order Taylor Series Method
	\item Fourth Order Runge-Kutta Method
\end{itemize}

These different methods vary in accuracy due to the error terms inherent in each technique.

The problem exists in finding the most accurate method (the method which produces the least error) in its scientific computing implementation.

A library for a family of single ode solvers will be implemented for comparison and analysis.


%Comment
%\wss{comment}

%You can also leave comments for yourself, like this:

%\an{comment}

\end{document}