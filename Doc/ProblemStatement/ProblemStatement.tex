\documentclass{article}

\usepackage{tabularx}
\usepackage{booktabs}

\title{CAS 741: Problem Statement\\Analysis and Benchmarking of a Family of Non-commercial ODE Solvers}

\author{Paul Aoanan - 000952218}

\date{September 15, 2017}

\input{../Comments}

\begin{document}

\maketitle

\begin{table}[hp]
\caption{Revision History} \label{TblRevisionHistory}
\begin{tabularx}{\textwidth}{llX}
\toprule
\textbf{Date} & \textbf{Developer(s)} & \textbf{Change}\\
\midrule
September 15, 2017 & Paul Aoanan & Initial Draft\\
%Date2 & Name(s) & Description of changes\\
%... & ... & ...\\
\bottomrule
\end{tabularx}
\end{table}

Known elementary techniques of solving ordinary differential equations (ODEs) use the linear
approximation method wherein the solution is based upon assuming or ``approximating" the slope
of the tangent line from one reference point to the next until the target point has been
reached.

The goal is to find a step-size so small that the approximate tangent line coincides
with the true curve.

Since these methods only yield approximate solutions, these methods introduce an error term to
the solution.

As well, the use of a small, near infinitesimal, step-size in machine computing can lead to a
large number of iterations and the introduction of round-off errors. Alternatively, the use of a
large step size can miss the goal of approximating the true curve.

Multiple methods in solving ODEs exist.
Examples of which are the following:
\begin{itemize}
	\item Euler's Method
	\item Heun's Method (Modified Euler's Method)
	\item Fourth Order Taylor Series Method
	\item Fourth Order Order Runge-Kutta Method
\end{itemize}

These different methods vary in accuracy due to the error terms inherent in each technique.

The problem exists in finding the most accurate method (the method which produces the least error) in its scientific computing implementation.


%Comment
%\wss{comment}

%You can also leave comments for yourself, like this:

%\an{comment}

\end{document}