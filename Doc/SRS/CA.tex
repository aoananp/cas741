\documentclass[12pt]{article}

\usepackage{amsmath, mathtools}
\usepackage{amsfonts}
\usepackage{amssymb}
\usepackage{graphicx}
\usepackage{colortbl}
\usepackage{xr}
\usepackage{hyperref}
\usepackage{longtable}
\usepackage{xfrac}
\usepackage{tabularx}
\usepackage{float}
\usepackage{siunitx}
\usepackage{booktabs}
\usepackage{caption}
\usepackage{pdflscape}
\usepackage{afterpage}

\usepackage[round]{natbib}

%\usepackage{refcheck}

\hypersetup{
    bookmarks=true,         % show bookmarks bar?
      colorlinks=true,       % false: boxed links; true: colored links
    linkcolor=red,          % color of internal links (change box color with linkbordercolor)
    citecolor=green,        % color of links to bibliography
    filecolor=magenta,      % color of file links
    urlcolor=cyan           % color of external links
}

%% Comments

\usepackage{color}

\newif\ifcomments\commentstrue

\ifcomments
\newcommand{\authornote}[3]{\textcolor{#1}{[#3 ---#2]}}
\newcommand{\todo}[1]{\textcolor{red}{[TODO: #1]}}
\else
\newcommand{\authornote}[3]{}
\newcommand{\todo}[1]{}
\fi

\newcommand{\wss}[1]{\authornote{blue}{SS}{#1}}
\newcommand{\wpa}[1]{\authornote{magenta}{PA}{#1}}


% For easy change of table widths
\newcommand{\colZwidth}{1.0\textwidth}
\newcommand{\colAwidth}{0.13\textwidth}
\newcommand{\colBwidth}{0.82\textwidth}
\newcommand{\colCwidth}{0.1\textwidth}
\newcommand{\colDwidth}{0.05\textwidth}
\newcommand{\colEwidth}{0.8\textwidth}
\newcommand{\colFwidth}{0.17\textwidth}
\newcommand{\colGwidth}{0.5\textwidth}
\newcommand{\colHwidth}{0.28\textwidth}

% Used so that cross-references have a meaningful prefix
\newcounter{defnum} %Definition Number
\newcommand{\dthedefnum}{GD\thedefnum}
\newcommand{\dref}[1]{GD\ref{#1}}
\newcounter{datadefnum} %Datadefinition Number
\newcommand{\ddthedatadefnum}{DD\thedatadefnum}
\newcommand{\ddref}[1]{DD\ref{#1}}
\newcounter{theorynum} %Theory Number
\newcommand{\tthetheorynum}{T\thetheorynum}
\newcommand{\tref}[1]{T\ref{#1}}
\newcounter{tablenum} %Table Number
\newcommand{\tbthetablenum}{T\thetablenum}
\newcommand{\tbref}[1]{TB\ref{#1}}
\newcounter{assumpnum} %Assumption Number
\newcommand{\atheassumpnum}{P\theassumpnum}
\newcommand{\aref}[1]{A\ref{#1}}
\newcounter{goalnum} %Goal Number
\newcommand{\gthegoalnum}{P\thegoalnum}
\newcommand{\gsref}[1]{GS\ref{#1}}
\newcounter{instnum} %Instance Number
\newcommand{\itheinstnum}{IM\theinstnum}
\newcommand{\iref}[1]{IM\ref{#1}}
\newcounter{reqnum} %Requirement Number
\newcommand{\rthereqnum}{P\thereqnum}
\newcommand{\rref}[1]{R\ref{#1}}
\newcounter{lcnum} %Likely change number
\newcommand{\lthelcnum}{LC\thelcnum}
\newcommand{\lcref}[1]{LC\ref{#1}}

\newcommand{\famname}{LODES} % PUT YOUR PROGRAM NAME HERE
\newcommand{\famdesc}{Library of ODE Solvers}

\usepackage{fullpage}

\begin{document}

\title{Commonality Analysis for a \famdesc{} (\famname{})} 
\author{Paul Aoanan}
\date{\today}

\maketitle

~\newpage

\pagenumbering{roman}

\section{Revision History}

\begin{tabularx}{\textwidth}{p{3cm}p{2cm}X}
\toprule {\bf Date} & {\bf Version} & {\bf Notes}\\
\midrule
\today & 1.0 & Initial Draft.\\
%Date 2 & 1.1 & Notes\\
\bottomrule
\end{tabularx}

~\newpage
	
\section{Reference Material}

This section records information for easy reference.

\subsection{Table of Units}

This section does not apply to this program family.

\subsection{Table of Symbols}

The table that follows summarizes the symbols used in this document along with
their units.  The choice of symbols was made to be consistent with the numeral analysis
and ordinary differential equation literature and with existing documentation
for solving ordinary differential equations.  The symbols are listed in alphabetical order.

\renewcommand{\arraystretch}{1.2}
%\noindent \begin{tabularx}{1.0\textwidth}{l l X}
\noindent \begin{longtable*}{l l p{12cm}} \toprule
\textbf{symbol} & \textbf{unit} & \textbf{description}\\
\midrule
$dy/dx$ & \text {-} & rate of change of $y$ depending on $x$\\
$f(x, y)$ & \text{-} & ODE function containing $(x,y)$\\
$h$ & \text{-} & step-size from $x_\text{(0)}$ to the next point $x_\text{(1)}$, where $x_\text{(1)} = x_\text{(0)} + h$\\
$n$ & \text{-} & reference recursion step.\\
$x_\text{0}$ & \text{-} & Initial value $x$\\
$x_\text{k}$ & \text{-} & Final value $x$\\
$x_\text{n}$ & \text{-} & Intermediate $n^\text{th}$ value $x$\\
$y_\text{0}$ & \text{-} & Initial value $y$\\ 
$y_\text{k}$ & \text{-} & Final value $y$\\
$y_\text{n}$ & \text{-} & Intermediate $n^\text{th}$ value $y$\\ 
$y'$ & \text{-} & first order ODE = $f(x, y)$\\
$y^\text{(n)}$ & \text{-} & ODE to the $n^\text{th}$ order\\
\bottomrule
\end{longtable*}

\subsection{Abbreviations and Acronyms}

\renewcommand{\arraystretch}{1.2}
\begin{tabular}{l l} 
  \toprule		
  \textbf{symbol} & \textbf{description}\\
  \midrule 
  A & Assumption\\
  DD & Data Definition\\
  GD & General Definition\\
  GS & Goal Statement\\
  IM & Instance Model\\
  LC & Likely Change\\
  ODE & Ordinary Differential Equation\\
  PS & Physical System Description\\
  R & Requirement\\
  SRS & Software Requirements Specification\\
  \famname{} & \famdesc{}\\
  T & Theoretical Model\\
  \bottomrule
\end{tabular}\\

\newpage

\tableofcontents

~\newpage

\pagenumbering{arabic}

\section{Introduction}

%\wss{This CA template is based on \citet{Smith2006}.

In physical sciences, mathematical models are derived from scientific models to
represent a real world phenomenon through formal mathematical constructs.

Scientific models in the study of radioactivity, carbon decay, and Newton's Law of Cooling
involve the use of ordinary differential equations (ODEs).

Known elementary techniques of solving ODEs in the discrete domain use the linear approximation
method wherein the solution is based upon assuming or ``approximating" the slope of the tangent
line from one reference point to the next until the target point has been reached.

The following section provides an overview of the Commonality Analysis (CA) for a program family of ODE solvers. The developed program will be
called \famdesc{} (\famname{}). This section explains the purpose of this
document,the scope of the family, and the characteristics of the intended readers
and the organization of the document.

\subsection{Purpose of Document}
The main purpose of this document is to formally describe program families of
the known well-known methods of solving ODEs. The goals and mathematical models used
in the \famname{} code are provided with
an emphasis on explicitly identifying assumptions, constraints, and unambiguous definitions.

This document contains the description of the functionalities of the \famname{} software
library. This document leads is the starting point for the subsequent software development
activities, including writing the requirements specification, design specification, code, and
the software verification and validation plan and execution.

\subsection{Scope of the Family}
The scope of the yamilu is limited to the the library of ODE solvers. Given
the appropriate inputs, each program in \famname{} is intended to find the solution to an
ODE problem.

\subsection{Characteristics of Intended Reader}
Reviewers of this document should have an elementary understanding of ordinary differential
equations and numerical methods, as typically covered in first and second year Calculus courses.
The users of \famname{} can have a lower level expertise, as explained in
Section~\ref{SecUserCharacteristics}.

\subsection{Organization of Document}
The organization of this document follows the template for a CA for scientific
computing software proposed by Smith (2006). The presentation follows the standard
pattern of presenting goals, theories, definitions, and assumptions. For readers that would
like a more bottom up approach, they can start reading the instance models in Section
\ref{sec_instance} and trace back to find any additional information they require.  The
instance models provide the methods to solve Ordinary Differential Equations (ODEs).

\section{General System Description}

This section identifies the interfaces between the system and its environment,
describes the potential user characteristics and lists the potential system
constraints.

\subsection{Potential System Contexts}

Figure~\ref{Fig_SystemContext} shows the system context.  A circle represents an
external entity outside the software, the user in this case.  A rectangle
represents the software system itself (\famname{}).  Arrows are used to show the data
flow between the system and its environment.

Programs in \famname{} are used inside a wrapper program.  The external interaction is through
program calls. The solution to the ODE is the output of the function.  The responsibilities of
the user and the system are as follows:

\begin{itemize}
\item User Responsibilities:
\begin{itemize}
\item Provide the correct program call, while adhering to conventions of the program's prototype
\item Provide the input details of the ODE to be solved, ensuring no errors in data entry
\item Declaration of the ODE method to be used in solving the ODE
\end{itemize}
\item \famname{} Responsibilities:
\begin{itemize}
\item Detect an improper input, such as invalid characters in the ODE statement and incomplete input arguments
\item Detect a data type mismatch where applicable, such as a string of characters in a floating point argument
\item Calculate the solution to the ODE problem
\end{itemize}
\end{itemize}

\subsection{Potential User Characteristics} \label{SecUserCharacteristics}

The end user of \famname{} should have an understanding of undergraduate Level
1 Calculus.

\subsection{Potential System Constraints}

There are no system constraints applicable.

\section{Commonalities}
This section first presents the background and motivation of the program family, which gives a
high-level view of the problem to be solved.  This is followed by the solution characteristics
specification, which presents the theories, definitions, assumptions, and finally
the instance models as variabilities.

\subsection{Background Overview} \label{Sec_Background}
\famname{} is a software library developed to be provide a means to solve ODE problems
using numerical methods. It can be used to solve different variations of ODEs given their
initial values. It can be implemented to find the most accurate method (the method which
produces the least error in their scientific computing implementation).

\subsection{Terminology and  Definitions}

This subsection provides a list of terms that are used in the subsequent
sections and their meaning, with the purpose of reducing ambiguity and making it
easier to correctly understand the requirements:

\begin{itemize}

\item Initial values: A "starting point" ($x_\text{0}, y_\text{0}$) of known values that
exists in the domain of the solution

\item Final values: An "ending point" ($x_\text{k}, y_\text{k}$) of unknown value $y_\text{k}$
that exists in the domain of the solution

\item Step size: The measure of arbitrary positive value ($h$) from the starting point
($x_\text{0}, y_\text{0}$) of the domain to the next ($x_\text{1}, y_\text{1}$), where
$x_\text{1} = x_\text{0} + h$

\item Derivative: The amount by which a a function changes at any given point as an
instantaneous rate of change

\item Numerical Analysis: A branch of mathematics and computer science wherein the solutions
are numerical approximations taking into account the errors involved in the process.

\item Numerical Approximation

\item Recursion:

\end{itemize}

\subsection{Data Definitions} \label{sec_datadef}

This section collects and defines all the data needed to build the instance
models. The dimension of each quantity is also given.  \wss{Modify the examples
  below for your problem, and add additional definitions as appropriate.}

~\newline

\noindent
\begin{minipage}{\textwidth}
\renewcommand*{\arraystretch}{1.5}
\begin{tabular}{| p{\colAwidth} | p{\colBwidth}|}
\hline
\rowcolor[gray]{0.9}
Number& DD\refstepcounter{datadefnum}\thedatadefnum \label{D_SLOPE}\\
\hline
Label& \bf Slope of the Tangent Line to a Curve\\
\hline
Symbol &$dy/dx$\\
\hline
% Units& $Mt^{-3}$\\
% \hline
  %Equation&$\frac{dy}{dx} = (y_{n+1} - y_n) / (x_{n+1} - x_n)$, of a function $y(x)$\\
  Equation&$dy/dx = $ $\lim_{h\to{0}} \frac{y(x+h) - y(x)}{h}$\\
  \hline
  Description 
        &$dy/dx$ is the slope of the tangent line to $y(x)$.\\
        &$h$ is the step size.\\
        &$n$ is the reference recursion step.\\
  \hline
  Source&
        Nagle, et al, "Solutions and Initial Value Problems," in
        \textit{Fundamentals of Differential Equations and Boundary Value Problems},
        3rd ed. USA: Addison Wesley Longman, 2000, ch. 1, p. 7.
  \\
  \hline
  Ref.\ By & \iref{ewat}\\
  \hline
\end{tabular}
\end{minipage}\\

\subsection{Goal Statements}

\noindent Given the required inputs, the goal statements are:

\begin{itemize}

\item[GS\refstepcounter{goalnum}\thegoalnum \label{G_SolveForY}:]{
Given an ordinary differential equation (ODE) represented by $y'= f(x,y)$, the set of initial
values $x_\text{0}$ and $y_\text{0}$ that satisfy $y(x_\text{0}) = y_\text{0}$,
and $x_\text{k}$, return $y_\text{k}$ such that $y(x_\text{k}) = y_\text{k}$ (the final
values), where $y(x)$ is a function, $f(x,y)$ is a function, and $x$
is an independent variable.}

\item[GS\refstepcounter{goalnum}\thegoalnum \label{G_InputOutput}:]{
Provide the user the means of providing the required inputs, calling the ODE solver
program, and displaying the results.}

\end{itemize}

\subsection{Theoretical Models} \label{sec_theoretical}

This section focuses on the general equations and laws that \famname{} is based
on.

~\newline

\noindent
\begin{minipage}{\textwidth}
\renewcommand*{\arraystretch}{1.5}
\begin{tabular}{| p{\colAwidth} | p{\colBwidth}|}
  \hline
  \rowcolor[gray]{0.9}
  Number& T\refstepcounter{theorynum}\thetheorynum \label{T_ODE}\\
  \hline
  Label&\bf Ordinary Differential Equation\\
  \hline
  Equation&  $y' = f(x,y)$\\
  %Equation&  $y^{(n)} = F(x,y,y',...,y^{(n-1)})$\\ %; y(x_\text{0}) = y_\text{0}$\\
  \hline
  Description & 
                The above model gives the definition of an ordinary differential equation.

                A differential equation is an equation, where the unknown is a
                function and both the function and its derivatives (rate of change) appear in the
                equation.

                An ordinary differential equation is a differential equation involving only ordinary derivatives
                with respect to a single independent variable.

                For an arbitrary ODE, the true solution will, in general, be unknown.

                Numerical methods are used to find numerical approximations of the solution to the ODE. 
                \\
  \hline
  Source &
           \url{http://users.math.msu.edu/users/gnagy/teaching/ode.pdf}\\
  % The above web link should be replaced with a proper citation to a publication
  \hline
  Ref.\ By & \tref{T_IVP}\\ %, \iref{euler}, \iref{heun}\\
  \hline
\end{tabular}
\end{minipage}\\

~\newline

\noindent
\begin{minipage}{\textwidth}
\renewcommand*{\arraystretch}{1.5}
\begin{tabular}{| p{\colAwidth} | p{\colBwidth}|}
  \hline
  \rowcolor[gray]{0.9}
  Number& T\refstepcounter{theorynum}\thetheorynum \label{T_IVP}\\
  \hline
  Label&\bf Existence and Uniqueness of the Solution\\
  \hline
  Equations&  $y' = f(x,y)$ [\tref{T_ODE}]
  ~\newline
  $y(x_\text{0}) = y_\text{0}$
  ~\newline
  Assuming $f$ and $y'$ are continuous in $R = \{(x,y): a < x < b, c < y < d\}$, where R is a rectangle
  and $a, b, c, d$ are its vertices
  ~\newline
  The initial value problem has a unique solution in some interval
  ~\newline
  $x_\text{0} - h < x < x_\text{0} + h$\\
  %Equation&  $y^{(n)} = F(x,y,y',...,y^{(n-1)})$\\ %; y(x_\text{0}) = y_\text{0}$\\
  \hline
  Description & 
                The above theoretical model shows that when an equation satisfies the initial values,
                it is assured that a solution to the initial value problem exists. It is desirable to know
                whether or not the equation has an existing solution before effort is made to solve it.
                As well, the theoretical model states that if a solution is found, then it is the only solution to
                the initial value problem. 
                \\
  \hline
  Source &
           Nagle, et al, "Solutions and Initial Value Problems," in
           \textit{Fundamentals of Differential Equations and Boundary Value Problems},
           3rd ed. USA: Addison Wesley Longman, 2000, ch. 1, p. 12.\\
  % The above web link should be replaced with a proper citation to a publication
  \hline
  Ref.\ By & \\%\iref{euler}, \iref{heun}\\
  \hline
\end{tabular}
\end{minipage}\\

~\newline

\section{Variabilities}
This section presents the variabilities in \famname{}. It details the varying instance models,
gives the assumptions for the input, the variabilities in the calculations, and
finally the target output.

\subsection{Assumptions}

\begin{itemize}

\item[A\refstepcounter{assumpnum}\theassumpnum \label{A_meaningfulLabel}:]
  \wss{Short description of each assumption.  Each assumption
    should have a meaningful label.  Use cross-references to identify the
    appropriate traceability to T, GD, DD etc., using commands like dref, ddref etc.}

\end{itemize}

\subsection{Calculation} \label{sec_Calculation}

\subsection{Output} \label{sec_Output}    

\section{Traceability Matrices and Graphs}

\wss{You will have to add tables.}

\newpage

\bibliographystyle {plainnat}
\bibliography {../../ReferenceMaterial/References}

\newpage

\section{Appendix}

\wss{Your report may require an appendix.  For instance, this is a good point to
show the values of the symbolic parameters introduced in the report.}

\subsection{Symbolic Parameters}

\wss{The definition of the requirements will likely call for SYMBOLIC\_CONSTANTS.
Their values are defined in this section for easy maintenance.}

\end{document}
