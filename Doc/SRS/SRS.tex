\documentclass[12pt]{article}

\usepackage{amsmath, mathtools}
\usepackage{amsfonts}
\usepackage{amssymb}
\usepackage{graphicx}
\usepackage{colortbl}
\usepackage{xr}
\usepackage{hyperref}
\usepackage{longtable}
\usepackage{xfrac}
\usepackage{tabularx}
\usepackage{float}
\usepackage{siunitx}
\usepackage{booktabs}
\usepackage{caption}
\usepackage{pdflscape}
\usepackage{afterpage}

\usepackage[round]{natbib}

%\usepackage{refcheck}

\hypersetup{
    bookmarks=true,         % show bookmarks bar?
      colorlinks=true,       % false: boxed links; true: colored links
    linkcolor=red,          % color of internal links (change box color with linkbordercolor)
    citecolor=green,        % color of links to bibliography
    filecolor=magenta,      % color of file links
    urlcolor=cyan           % color of external links
}

%% Comments

\usepackage{color}

\newif\ifcomments\commentstrue

\ifcomments
\newcommand{\authornote}[3]{\textcolor{#1}{[#3 ---#2]}}
\newcommand{\todo}[1]{\textcolor{red}{[TODO: #1]}}
\else
\newcommand{\authornote}[3]{}
\newcommand{\todo}[1]{}
\fi

\newcommand{\wss}[1]{\authornote{blue}{SS}{#1}}
\newcommand{\wpa}[1]{\authornote{magenta}{PA}{#1}}


% For easy change of table widths
\newcommand{\colZwidth}{1.0\textwidth}
\newcommand{\colAwidth}{0.13\textwidth}
\newcommand{\colBwidth}{0.82\textwidth}
\newcommand{\colCwidth}{0.1\textwidth}
\newcommand{\colDwidth}{0.05\textwidth}
\newcommand{\colEwidth}{0.8\textwidth}
\newcommand{\colFwidth}{0.17\textwidth}
\newcommand{\colGwidth}{0.5\textwidth}
\newcommand{\colHwidth}{0.28\textwidth}

% Used so that cross-references have a meaningful prefix
\newcounter{defnum} %Definition Number
\newcommand{\dthedefnum}{GD\thedefnum}
\newcommand{\dref}[1]{GD\ref{#1}}
\newcounter{datadefnum} %Datadefinition Number
\newcommand{\ddthedatadefnum}{DD\thedatadefnum}
\newcommand{\ddref}[1]{DD\ref{#1}}
\newcounter{theorynum} %Theory Number
\newcommand{\tthetheorynum}{T\thetheorynum}
\newcommand{\tref}[1]{T\ref{#1}}
\newcounter{tablenum} %Table Number
\newcommand{\tbthetablenum}{T\thetablenum}
\newcommand{\tbref}[1]{TB\ref{#1}}
\newcounter{assumpnum} %Assumption Number
\newcommand{\atheassumpnum}{P\theassumpnum}
\newcommand{\aref}[1]{A\ref{#1}}
\newcounter{goalnum} %Goal Number
\newcommand{\gthegoalnum}{P\thegoalnum}
\newcommand{\gsref}[1]{GS\ref{#1}}
\newcounter{instnum} %Instance Number
\newcommand{\itheinstnum}{IM\theinstnum}
\newcommand{\iref}[1]{IM\ref{#1}}
\newcounter{reqnum} %Requirement Number
\newcommand{\rthereqnum}{P\thereqnum}
\newcommand{\rref}[1]{R\ref{#1}}
\newcounter{lcnum} %Likely change number
\newcommand{\lthelcnum}{LC\thelcnum}
\newcommand{\lcref}[1]{LC\ref{#1}}

\newcommand{\progname}{LODES} % PUT YOUR PROGRAM NAME HERE
\newcommand{\progdesc}{Family of ODE Solvers}

\usepackage{fullpage}

\begin{document}

\title{Commonality Analysis / Software Requirements Specification for a \progdesc{} (\progname{})} 
\author{Paul Aoanan}
\date{\today}
	
\maketitle

\pagenumbering{roman}
\tableofcontents

\begin{table}[bp]
\caption{\bf Revision History}
\begin{tabularx}{\textwidth}{p{3cm}p{2cm}X}
\toprule {\bf Date} & {\bf Version} & {\bf Notes}\\
\midrule
\today & 1.0 & First draft.\\
%Date 2 & 1.1 & Notes\\
\bottomrule
\end{tabularx}
\end{table}

\section{Reference Material}

This section records information for easy reference.

\subsection{Table of Units}

This section is not relevant to this CA/SRS.

\subsection{Table of Symbols}

The table that follows summarizes the symbols used in this document along with
their units.  The choice of symbols was made to be consistent with the numeral analysis
and ordinary differential equation literature and with existing documentation
for solving ordinary differential equations.  The symbols are listed in alphabetical order.

\renewcommand{\arraystretch}{1.2}
%\noindent \begin{tabularx}{1.0\textwidth}{l l X}
\noindent \begin{longtable*}{l l p{12cm}} \toprule
\textbf{symbol} & \textbf{unit} & \textbf{description}\\
\midrule
$dy/dx$ & \text {-} & rate of change of $y$ depending on $x$\\
$f(x, y)$ & \text{-} & ODE function containing $(x,y)$\\
$h$ & \text{-} & step-size from $x_\text{(0)}$ to the next point $x_\text{(1)}$, where $x_\text{(1)} = x_\text{(0)} + h$\\
$n$ & \text{-} & reference recursion step.\\
$x_\text{0}$ & \text{-} & Initial value $x$\\
$x_\text{k}$ & \text{-} & Final value $x$\\
$x_\text{n}$ & \text{-} & Intermediate $n^\text{th}$ value $x$\\
$y_\text{0}$ & \text{-} & Initial value $y$\\ 
$y_\text{k}$ & \text{-} & Final value $y$\\
$y_\text{n}$ & \text{-} & Intermediate $n^\text{th}$ value $y$\\ 
$y'$ & \text{-} & first order ODE = $f(x, y)$\\
$y^\text{(n)}$ & \text{-} & ODE to the $n^\text{th}$ order\\
\bottomrule
\end{longtable*}

\subsection{Abbreviations and Acronyms}

\renewcommand{\arraystretch}{1.2}
\begin{tabular}{l l} 
  \toprule		
  \textbf{symbol} & \textbf{description}\\
  \midrule 
  A & Assumption\\

  DD & Data Definition\\
  GD & General Definition\\
  GS & Goal Statement\\
  IM & Instance Model\\
  LC & Likely Change\\
  ODE & Ordinary Differential Equation\\
  PS & Physical System Description\\
  R & Requirement\\
  SRS & Software Requirements Specification\\
  \progname{} & \progdesc{}\\
  T & Theoretical Model\\
  \bottomrule
\end{tabular}\\

\wss{Add any other abbreviations or acronyms that you add}

\newpage
\pagenumbering{arabic}

\section{Introduction}

In physical sciences, mathematical models are derived from scientific models to
represent a real world phenomenon through formal mathematical constructs.

Scientific models in the study of radioactivity, carbon decay, and Newton's Law of Cooling
involve the use of ordinary differential equations (ODEs).

Known elementary techniques of solving ODEs in the discrete domain use the linear approximation
method wherein the solution is based upon assuming or ``approximating" the slope of the tangent
line from one reference point to the next until the target point has been reached.

The following section provides an overview of the Commonality Analysis (CA) for a program family of ODE solvers. The developed program will be
called Library of ODE Solvers (\progname{}). This section explains the purpose of this
document,the scope of the system, and the characteristics of the intended readers
and the organization of the document.

\subsection{Purpose of Document}

The main purpose of this document is to formally describe program families of
the known well-known methods of solving ODEs. The goals and mathematical models used
in the \progname{} code are provided with
an emphasis on explicitly identifying assumptions, constraints, and unambiguous definitions.

This document contains the description of the functionalities of the \progname{} software library as well
as the non-functional requirements that the software may have to meet. This document leads is the
starting point for the
subsequent software development activities, including writing the requirements specification,
design specification, code, and the software verification and validation plan and execution.

\subsection{Scope of Requirements} 

The scope of requirements is limited to the analysis of the library of ODE solvers. Given
the appropriate inputs, each program in \progname{} is intended to find the solution to an
ODE problem.

\subsection{Characteristics of Intended Reader}

Reviewers of this document should have an elementary understanding of ordinary differential
equations and numerical methods, as typically covered in first and second year Calculus courses.
The users of \progname{} can have a lower level expertise, as explained in
Section~\ref{SecUserCharacteristics}.

\subsection{Organization of Document}

The organization of this document follows the template for a CA for scientific
computing software proposed by Smith (2006). The presentation follows the standard
pattern of presenting goals, theories, definitions, and assumptions. For readers that would
like a more bottom up approach, they can start reading the instance models in Section
\ref{sec_instance} and trace back to find any additional information they require.  The
instance models provide the methods to solve Ordinary Differential Equations (ODEs).

\section{General System Description}

This section identifies the interfaces between the system and its environment,
describes the user characteristics and lists the system constraints.

\subsection{Potential System Contexts}

Figure~\ref{Fig_SystemContext} shows the system context.  A circle represents an
external entity outside the software, the user in this case.  A rectangle
represents the software system itself (\progname{}).  Arrows are used to show the data
flow between the system and its environment.

Programs in \progname{} are used inside a wrapper program.  The external interaction is through
program calls. The solution to the ODE is the output of the function.  The responsibilities of
the user and the system are as follows:

\begin{itemize}
\item User Responsibilities:
\begin{itemize}
\item Provide the correct program call, while adhering to conventions of the program's prototype
\item Provide the input details of the ODE to be solved, ensuring no errors in data entry
\item Declaration of the ODE method to be used in solving the ODE
\end{itemize}
\item \progname{} Responsibilities:
\begin{itemize}
\item Detect an improper input, such as invalid characters in the ODE statement and incomplete input arguments
\item Detect a data type mismatch where applicable, such as a string of characters in a floating point argument
\item Calculate the solution to the ODE problem
\end{itemize}
\end{itemize}

\subsection{Potential User Characteristics} \label{SecUserCharacteristics}

The end user of \progname{} should have an understanding of undergraduate Level
1 Calculus.

\subsection{Potential System Constraints}

There are no system constraints.

\section{Commonalities}

This section first presents the background and motivation of the program family, which gives a high-level
view of the problem to be solved.  This is followed by the solution characteristics
specification, which presents the theories, definitions, assumptions, and finally
the instance models as variabilities.

\subsection{Background Overview} \label{Sec_bo}

\progname{} is a software library developed to be provide a means to solve ODE problems
using numerical methods. It can be used to solve different variations of ODEs given their initial values.
It can be implemented to find the most accurate method (the method which produces the least error in their
scientific computing implementation).

\subsection{Terminology and Definitions}

This subsection provides a list of terms that are used in the subsequent
sections and their meaning, with the purpose of reducing ambiguity and making it
easier to correctly understand the requirements:

\begin{itemize}

\item Initial values: A "starting point" ($x_\text{0}, y_\text{0}$) of known values that
exists in the domain of the solution

\item Final values: An "ending point" ($x_\text{k}, y_\text{k}$) of unknown value $y_\text{k}$
that exists in the domain of the solution

\item Step size: The measure of arbitrary positive value ($h$) from the starting point
($x_\text{0}, y_\text{0}$) of the domain to the next ($x_\text{1}, y_\text{1}$), where
$x_\text{1} = x_\text{0} + h$

\item Derivative: The amount by which a a function changes at any given point as an instantaneous
rate of change

\item Numerical Analysis: A branch of mathematics and computer science wherein the solutions
are numerical approximations taking into account the errors involved in the process.

\item Recursion: 

%\item Numerical Approximation: 

\end{itemize}

\subsection{Goal Statements} \label{Sec_gs}

\begin{itemize}

\item[GS\refstepcounter{goalnum}\thegoalnum \label{G_meaningfulLabel}:]{
Given an ordinary differential equation (ODE) represented by $y'= f(x,y)$, the set of initial values
$x_\text{0}$ and $y_\text{0}$ that satisfy $y(x_\text{0}) = y_\text{0}$, and $x_\text{k}$, return $y_\text{k}$
such that $y(x_\text{k}) = y_\text{k}$ (the final values), where $y(x)$ is a function, $f(x,y)$ is a function,
and $x$ is an independent variable.}

\item[GS\refstepcounter{goalnum}\thegoalnum \label{G_meaningfulLabel}:]{
Provide the user the means of providing the required inputs, calling the ODE solver program, and displaying the results.
}

\end{itemize}

\subsection{Theoretical Models}\label{sec_theoretical}

This section focuses on the general equations and laws that \progname{} is based
on.

~\newline

\noindent
\begin{minipage}{\textwidth}
\renewcommand*{\arraystretch}{1.5}
\begin{tabular}{| p{\colAwidth} | p{\colBwidth}|}
  \hline
  \rowcolor[gray]{0.9}
  Number& T\refstepcounter{theorynum}\thetheorynum \label{T_ODE}\\
  \hline
  Label&\bf Ordinary Differential Equation\\
  \hline
  Equation&  $y' = f(x,y)$\\
  %Equation&  $y^{(n)} = F(x,y,y',...,y^{(n-1)})$\\ %; y(x_\text{0}) = y_\text{0}$\\
  \hline
  Description & 
                The above model gives the definition of an ordinary differential equation.

                A differential equation is an equation, where the unknown is a
                function and both the function and its derivatives (rate of change) appear in the
                equation.

                An ordinary differential equation is a differential equation involving only ordinary derivatives
                with respect to a single independent variable.

                For an arbitrary ODE, the true solution will, in general, be unknown.

                Numerical methods are used to find numerical approximations of the solution to the ODE. 
                \\
  \hline
  Source &
           \url{http://users.math.msu.edu/users/gnagy/teaching/ode.pdf}\\
  % The above web link should be replaced with a proper citation to a publication
  \hline
  Ref.\ By & \tref{T_IVP}, \iref{euler}, \iref{heun}\\
  \hline
\end{tabular}
\end{minipage}\\

~\newline

\noindent
\begin{minipage}{\textwidth}
\renewcommand*{\arraystretch}{1.5}
\begin{tabular}{| p{\colAwidth} | p{\colBwidth}|}
  \hline
  \rowcolor[gray]{0.9}
  Number& T\refstepcounter{theorynum}\thetheorynum \label{T_IVP}\\
  \hline
  Label&\bf Existence and Uniqueness of the Solution\\
  \hline
  Equations&  $y' = f(x,y)$ [\tref{T_ODE}]
  ~\newline
  $y(x_\text{0}) = y_\text{0}$
  ~\newline
  Assuming $f$ and $y'$ are continuous in $R = \{(x,y): a < x < b, c < y < d\}$, where R is a rectangle
  and $a, b, c, d$ are its vertices
  ~\newline
  The initial value problem has a unique solution in some interval
  ~\newline
  $x_\text{0} - h < x < x_\text{0} + h$\\
  %Equation&  $y^{(n)} = F(x,y,y',...,y^{(n-1)})$\\ %; y(x_\text{0}) = y_\text{0}$\\
  \hline
  Description & 
                The above theoretical model shows that when an equation satisfies the initial values,
                it is assured that a solution to the initial value problem exists. It is desirable to know
                whether or not the equation has an existing solution before effort is made to solve it.
                As well, the theoretical model states that if a solution is found, then it is the only solution to
                the initial value problem. 
                \\
  \hline
  Source &
           Nagle, et al, "Solutions and Initial Value Problems," in
           \textit{Fundamentals of Differential Equations and Boundary Value Problems},
           3rd ed. USA: Addison Wesley Longman, 2000, ch. 1, p. 12.\\
  % The above web link should be replaced with a proper citation to a publication
  \hline
  Ref.\ By & \iref{euler}, \iref{heun}\\
  \hline
\end{tabular}
\end{minipage}\\

\subsection{General Definitions}\label{sec_gendef}

This section collects the laws and equations that will be used in deriving the
data definitions, which in turn are used to build the instance models.
This section does not apply to this program family.

\subsection{Data Definitions}\label{sec_datadef}

This section collects and defines all the data needed to build the instance
models. The dimension of each quantity is also given.
This section does not apply to this program family.
~\newline

% ~\newline

% \noindent
% \begin{minipage}{\textwidth}
% \renewcommand*{\arraystretch}{1.5}
% \begin{tabular}{| p{\colAwidth} | p{\colBwidth}|}
%   \hline
%   \rowcolor[gray]{0.9}
%   Number& T\refstepcounter{theorynum}\thetheorynum \label{T_SODE}\\
%   \hline
%   Label&\bf Higher Order Ordinary Differential Equation\\
%   \hline
%   Equations&
%     $y^{(n)} = f(x,y,y^\text{(1)},y^\text{(2)},...,y^\text{(n-1)})$
%     ~\newline
%     ~\newline
%     $f_\text{(1)} = y, f_\text{(2)} = y^\text{(1)},..., f_\text{(n)} = y^\text{(n-1)}$
%     ~\newline
%     ~\newline
%     \[
%         \left[
%           \begin{tabular}{c}
%            $f_\text{(n)}$\\
%            $f_\text{(n-1)}$\\
%            \text{.}\\
%            \text{.}\\
%            \text{.}\\
%            $f_\text{(2)}$\\
%            $f_\text{(1)}$
%           \end{tabular}
%         \right]
%     \]
%     \\
%   %Equation&  $y^{(n)} = F(x,y,y',...,y^{(n-1)})$\\ %; y(x_\text{0}) = y_\text{0}$\\
%   \hline
%   Description & 
%                 The above model provides a 
%                 \\
%   \hline
%   Source &
%            \url{https://www.siam.org/books/ot98/sample/OT98Chapter5.pdf}\\
%   % The above web link should be replaced with a proper citation to a publication
%   \hline
%   Ref.\ By & \tref{T_IVP}\\
%   \hline
% \end{tabular}
% \end{minipage}\\

~\newline

\section{Variabilities}

This section presents the variabilities in \progname{}. It details the varying instance models,
gives the assumptions for the input, the variabilities in the calculations, and
finally the target output.

\subsection{Instance Models} \label{sec_instance}    

This section transforms the problem defined in Section~\ref{Sec_bo} into 
one which is expressed in mathematical terms. It uses concrete symbols defined 
in Section~\ref{sec_theoretical}.

~\newline

%Instance Model 1

\noindent
\begin{minipage}{\textwidth}
\renewcommand*{\arraystretch}{1.5}
\begin{tabular}{| p{\colAwidth} | p{\colBwidth}|}
  \hline
  \rowcolor[gray]{0.9}
  Number& IM\refstepcounter{instnum}\theinstnum \label{euler}\\
  \hline
  Label& \bf Euler's Method of Finding the Solution to an ODE\\
  \hline
  Input& $y' = f(x,y), h, x_\text{0}, y_\text{0}, x_\text{k}$\\
  \hline
  Output& $y_\text{k}$ such that $y_\text{k} = y(x_\text{k})$  \\
  &using the recursive formulas:\\
  &$x_\text{n+1} = x_\text{n} + h, n = 0, 1, 2,...$\\
  &$y_\text{n+1} = y_\text{n} + h*f(x_\text{n}, y_\text{n}), n = 0, 1, 2,...$\\
  \hline
  Description&$y' = f(x, y)$ is the first order ODE.\\
  &$h$ is the constant step size.\\
  &$x_\text{0}$ is the initial value of $x$.\\
  &$x_\text{n+1}$ is the value of $x$ in the next equation iteration.\\
  &$x_\text{k}$ is the final value of $x$.\\
  &$y_\text{0}$ is the initial value of $y$, such that $y_\text{0} = y(x_\text{0})$.\\
  &$y_\text{n+1}$ is the value of $y$ in the next equation iteration.\\
  &$y_\text{k}$ is the final value of $y$.\\
  &$n$ is the reference recursion step.\\

  & The above equations are used recursively until $x_\text{n+1} = x_\text{k}$ and $y_\text{n+1} = y_\text{k}$.
  \\
  \hline
  Sources&
        Nagle, et al, "The Approximation Method of Euler," in
        \textit{Fundamentals of Differential Equations and Boundary Value Problems},
        3rd ed. USA: Addison Wesley Longman, 2000, ch. 1, sec. 1.5, pp. 31-32.
  \\
  \hline
  Ref.\ By & \iref{heun}\\
  \hline
\end{tabular}
\end{minipage}\\

\subsubsection*{Detailed derivation of Euler's Method}

Let $y' = f(x, y),$   $y(x_0) = y_0$, $h$ as a fixed positive number
(step-size), and consider the equally spaced points in the domain:\\
\\
\hspace*{2ex} $x_n = x_0 + nh,$   $n = 0, 1, 2...$\\
\\
The construction of values $y_n$ that approximate the solution values proceeds as follows.
At the point $(x_0, y_0)$, the slope of the solution to $y' = f(x, y)$ is given by
${dy}/{dx} = f(x_0, y_0)$. Hence the tangent line to the solution curve at the initial point is:\\
\\
\hspace*{2ex} $y = y_0 + (x - x_0)*f(x_0, y_0)$.\\
\\
Using this tangent line to approximate the next point in the solution set, we find that
for the point $x_1 = x_0 + h$, we can assume the following approximation:\\
\\
\hspace*{2ex} $y_1 = y_0 + h*f(x_0, y_0).$\\
\\
Repeating the process (recursion), until we reach $y_k$ yields the derivation of Euler's Method.

~\newline

%Instance Model 2

\noindent
\begin{minipage}{\textwidth}
\renewcommand*{\arraystretch}{1.5}
\begin{tabular}{| p{\colAwidth} | p{\colBwidth}|}
  \hline
  \rowcolor[gray]{0.9}
  Number& IM\refstepcounter{instnum}\theinstnum \label{heun}\\
  \hline
  Label& \bf Heun's Method of Finding the Solution to an ODE\\
  \hline
  Input& $y' = f(x,y), h, x_\text{0}, y_\text{0}, x_\text{k}$\\
  \hline
  Output& $y_\text{k}$ such that $y_\text{k} = y(x_\text{k})$  \\
  &using the recursive formulas:\\
  &$x_\text{n+1} = x_\text{n} + h, n = 0, 1, 2,...$\\
  &$y_\text{n+1} = y_\text{n} + \frac{h}{2}\{[f(x_n, y_n) + f(x_n + h, y_n + h*f(x_n, y_n)]\}, n = 0, 1, 2,...$\\
  \hline
  Description&$y' = f(x, y)$ is the first order ODE.\\
  &$h$ is the constant step size.\\
  &$x_\text{0}$ is the initial value of $x$.\\
  &$x_\text{n+1}$ is the value of $x$ in the next equation iteration.\\
  &$x_\text{k}$ is the final value of $x$.\\
  &$y_\text{0}$ is the initial value of $y$, such that $y_\text{0} = y(x_\text{0})$.\\
  &$y_\text{n+1}$ is the value of $y$ in the next equation iteration.\\
  &$y_\text{k}$ is the final value of $y$.\\
  &$n$ is the reference recursion step.\\

  & The above equations are used recursively until $x_\text{n+1} = x_\text{k}$ and $y_\text{n+1} = y_\text{k}$.
  \\
  \hline
  Sources&
        Nagle, et al, "Improved Euler's Method," in
        \textit{Fundamentals of Differential Equations and Boundary Value Problems},
        3rd ed. USA: Addison Wesley Longman, 2000, ch. 3, sec. 3.5, pp. 124-129.
  \\
  \hline
  Ref.\ By & \iref{epcm}\\
  \hline
\end{tabular}
\end{minipage}\\


\subsection{Input Assumptions} \label{Sec_ia}

This section focuses on the variabilities and assumptions in the inputs of \progname{}.

~\newline

\noindent
\begin{minipage}{\textwidth}
\renewcommand*{\arraystretch}{1.5}
\begin{tabular}{|p{0.35\textwidth}| p{0.60\textwidth}|}
  \hline
  \rowcolor[gray]{0.9}
  Input Variability& Parameter of Variation\\
  \hline
  Allowed program family calls & Set of \{Euler's Method, Heun's Method, Fourth-Order Taylor Series, Fourth-Order Runge-Kutta\}\\
  \hline
  Allowed order of $f(x,y)$ & Set of \{First\} \\
  \hline
  Allowed type of $f(x,y)$ & Set of \{Linear, Nonlinear, Homogeneous, Non-homogeneous\}\\
  \hline
  % Allowed dimensions of $f(x,y) = f(x,y)_\text{dim}$& set of $\mathbb{N}$\\
  Allowed dimensions of $f(x,y)$ & Set of \{Single\}\\ 
  \hline
  Possible entries of $h$ & set of positive $\mathbb{R}$\\
  \hline
  Possible entries of $x_\text{0}$ & set of $\mathbb{R}$\\
  \hline
  % Allowed dimensions of $x_\text{0} = x_\text{(0)dim}$& set of $\mathbb{N}$, where $x_\text{(0)dim} = f(x,y)_\text{dim}$\\
  Allowed dimensions of $x_\text{0}$ & Set of \{Single\}\\
  \hline
  Possible entries of $y_\text{0}$ & set of $\mathbb{R}$\\
  \hline
  % Allowed dimensions of $y_\text{0} = y_\text{(0)dim}$& set of $\mathbb{N}$, where $y_\text{(0)dim} = f(x,y)_\text{dim}$\\
  Allowed dimensions of $y_\text{0}$ & Set of \{Single\}\\
  \hline
  Possible entries of $x_\text{k}$& set of $\mathbb{R}$\\
  \hline
  Allowed dimensions of $x_\text{k}$ & Set of \{Single\}\\
  \hline
\end{tabular}
\end{minipage}\\

~\newline

\subsection{Calculation} \label{Sec_calc}

This section focuses on the variabilities in the calculations used in \progname{}.

~\newline

\noindent
\begin{minipage}{\textwidth}
\renewcommand*{\arraystretch}{1.5}
\begin{tabular}{|p{0.35\textwidth}| p{0.60\textwidth}|}
  \hline
  \rowcolor[gray]{0.9}
  Variability& Parameter of Variation\\
  \hline
  Check input? & boolean (false if the input is assumed to satisfy the input assumptions)\\

  \hline
\end{tabular}
\end{minipage}\\

~\newline

\subsection{Output} \label{Sec_calc}

This section focuses on the variabilities in the output of \progname{}.

~\newline

\noindent
\begin{minipage}{\textwidth}
\renewcommand*{\arraystretch}{1.5}
\begin{tabular}{|p{0.35\textwidth}| p{0.60\textwidth}|}
  \hline
  \rowcolor[gray]{0.9}
  Variability& Parameter of Variation\\
  \hline
  Destination for output $y_k$ & boolean set of \{to a file, to the screen, to memory\}\\
  \hline
  Encoding of output $y_k$& Set of {binary, text} \\
  \hline
  % Output residual & boolean (true if the program returns the residual)\\
  % \hline
  Possible values of $y_k$ & set of $\mathbb{R}$ $\cup$ $\{-\infty, \infty,$ undefined$\}$ \\ 
  \hline
  Output program success & boolean (true if the program successfully
  solves for the solution to the ODE problem)\\
  \hline
\end{tabular}
\end{minipage}\\

~\newline

% \subsection{Nonfunctional Requirements}

% \wss{List your nonfunctional requirements.  You may consider using a fit
%   criterion to make them verifiable.}

% \section{Likely Changes}    

% \noindent \begin{itemize}

% \item[LC\refstepcounter{lcnum}\thelcnum\label{LC_meaningfulLabel}:] \wss{Give
%     the likely changes, with a reference to the related assumption (aref), as appropriate.}

% \end{itemize}

% \section{Traceability Matrices and Graphs}

% The purpose of the traceability matrices is to provide easy references on what
% has to be additionally modified if a certain component is changed.  Every time a
% component is changed, the items in the column of that component that are marked
% with an ``X'' may have to be modified as well.  Table~\ref{Table:trace} shows the
% dependencies of theoretical models, general definitions, data definitions, and
% instance models with each other. Table~\ref{Table:R_trace} shows the
% dependencies of instance models, requirements, and data constraints on each
% other. Table~\ref{Table:A_trace} shows the dependencies of theoretical models,
% general definitions, data definitions, instance models, and likely changes on
% the assumptions.

% \wss{You will have to modify these tables for your problem.}

% \afterpage{
% \begin{landscape}
% \begin{table}[h!]
% \centering
% \begin{tabular}{|c|c|c|c|c|c|c|c|c|c|c|c|c|c|c|c|c|c|c|c|}
% \hline
% 	& \aref{A_OnlyThermalEnergy}& \aref{A_hcoeff}& \aref{A_mixed}& \aref{A_tpcm}& \aref{A_const_density}& \aref{A_const_C}& \aref{A_Newt_coil}& \aref{A_tcoil}& \aref{A_tlcoil}& \aref{A_Newt_pcm}& \aref{A_charge}& \aref{A_InitTemp}& \aref{A_OpRangePCM}& \aref{A_OpRange}& \aref{A_htank}& \aref{A_int_heat}& \aref{A_vpcm}& \aref{A_PCM_state}& \aref{A_Pressure} \\
% \hline
% \tref{T_COE}        & X& & & & & & & & & & & & & & & & & & \\ \hline
% \tref{T_SHE}        & & & & & & & & & & & & & & & & & & & \\ \hline
% \tref{T_LHE}        & & & & & & & & & & & & & & & & & & & \\ \hline
% \dref{NL}           & & X& & & & & & & & & & & & & & & & & \\ \hline
% \dref{ROCT}         & & & X& X& X& X& & & & & & & & & & & & & \\ \hline
% \ddref{FluxCoil}    & & & & & & & X& X& X& & & & & & & & & & \\ \hline
% \ddref{FluxPCM}     & & & X& X& & & & & & X& & & & & & & & & \\ \hline
% \ddref{D_HOF}       & & & & & & & & & & & & & & & & & & & \\ \hline
% \ddref{D_MF}        & & & & & & & & & & & & & & & & & & & \\ \hline
% \iref{ewat}         & & & & & & & & & & & X& X& & X& X& X& & & X \\ \hline
% \iref{epcm}         & & & & & & & & & & & & X& X& & & X& X& X& \\ \hline
% \iref{I_HWAT}       & & & & & & & & & & & & & & X& & & & & X \\ \hline
% \iref{I_HPCM}       & & & & & & & & & & & & & X& & & & & X & \\ \hline
% \lcref{LC_tpcm}     & & & & X& & & & & & & & & & & & & & & \\ \hline
% \lcref{LC_tcoil}    & & & & & & & & X& & & & & & & & & & & \\ \hline
% \lcref{LC_tlcoil}   & & & & & & & & & X& & & & & & & & & & \\ \hline
% \lcref{LC_charge}   & & & & & & & & & & & X& & & & & & & & \\ \hline
% \lcref{LC_InitTemp} & & & & & & & & & & & & X& & & & & & & \\ \hline
% \lcref{LC_htank}    & & & & & & & & & & & & & & & X& & & & \\
% \hline
% \end{tabular}
% \caption{Traceability Matrix Showing the Connections Between Assumptions and Other Items}
% \label{Table:A_trace}
% \end{table}
% \end{landscape}
% }

% \begin{table}[h!]
% \centering
% \begin{tabular}{|c|c|c|c|c|c|c|c|c|c|c|c|c|c|c|c|c|c|c|c|c|c|c|c|}
% \hline        
% 	& \tref{T_COE}& \tref{T_SHE}& \tref{T_LHE}& \dref{NL}& \dref{ROCT} & \ddref{FluxCoil}& \ddref{FluxPCM} & \ddref{D_HOF}& \ddref{D_MF}& \iref{ewat}& \iref{epcm}& \iref{I_HWAT}& \iref{I_HPCM} \\
% \hline
% \tref{T_COE}     & & & & & & & & & & & & & \\ \hline
% \tref{T_SHE}     & & & X& & & & & & & & & & \\ \hline
% \tref{T_LHE}     & & & & & & & & & & & & & \\ \hline
% \dref{NL}        & & & & & & & & & & & & & \\ \hline
% \dref{ROCT}      & X& & & & & & & & & & & & \\ \hline
% \ddref{FluxCoil} & & & & X& & & & & & & & & \\ \hline
% \ddref{FluxPCM}  & & & & X& & & & & & & & & \\ \hline
% \ddref{D_HOF}    & & & & & & & & & & & & & \\ \hline
% \ddref{D_MF}     & & & & & & & & X& & & & & \\ \hline
% \iref{ewat}      & & & & & X& X& X& & & & X& & \\ \hline
% \iref{epcm}      & & & & & X& & X& & X& X& & & X \\ \hline
% \iref{I_HWAT}    & & X& & & & & & & & & & & \\ \hline
% \iref{I_HPCM}    & & X& X& & & & X& X& X& & X& & \\
% \hline
% \end{tabular}
% \caption{Traceability Matrix Showing the Connections Between Items of Different Sections}
% \label{Table:trace}
% \end{table}

% \begin{table}[h!]
% \centering
% \begin{tabular}{|c|c|c|c|c|c|c|c|}
% \hline
% 	& \iref{ewat}& \iref{epcm}& \iref{I_HWAT}& \iref{I_HPCM}& \ref{sec_DataConstraints}& \rref{R_RawInputs}& \rref{R_MassInputs} \\
% \hline
% \iref{ewat}            & & X& & & & X& X \\ \hline
% \iref{epcm}            & X& & & X& & X& X \\ \hline
% \iref{I_HWAT}          & & & & & & X& X \\ \hline
% \iref{I_HPCM}          & & X& & & & X& X \\ \hline
% \rref{R_RawInputs}     & & & & & & & \\ \hline
% \rref{R_MassInputs}    & & & & & & X& \\ \hline
% \rref{R_CheckInputs}   & & & & & X& & \\ \hline
% \rref{R_OutputInputs}  & X& X& & & & X& X \\ \hline
% \rref{R_TempWater}     & X& & & & & & \\ \hline 
% \rref{R_TempPCM}       & & X& & & & & \\ \hline
% \rref{R_EnergyWater}   & & & X& & & & \\ \hline
% \rref{R_EnergyPCM}     & & & & X& & & \\ \hline
% \rref{R_VerifyOutput}  & & & X& X& & & \\ \hline
% \rref{R_timeMeltBegin} & & X& & & & & \\ \hline
% \rref{R_timeMeltEnd}   & & X& & & & & \\ 
% \hline
% \end{tabular}
% \caption{Traceability Matrix Showing the Connections Between Requirements and Instance Models}
% \label{Table:R_trace}
% \end{table}

% The purpose of the traceability graphs is also to provide easy references on
% what has to be additionally modified if a certain component is changed.  The
% arrows in the graphs represent dependencies. The component at the tail of an
% arrow is depended on by the component at the head of that arrow. Therefore, if a
% component is changed, the components that it points to should also be
% changed. Figure~\ref{Fig_ATrace} shows the dependencies of theoretical models,
% general definitions, data definitions, instance models, likely changes, and
% assumptions on each other. Figure~\ref{Fig_RTrace} shows the dependencies of
% instance models, requirements, and data constraints on each other.

% \begin{figure}[h!]
% 	\begin{center}
% 		%\rotatebox{-90}
% 		{
% 			\includegraphics[width=\textwidth]{ATrace.png}
% 		}
% 		\caption{\label{Fig_ATrace} Traceability Matrix Showing the Connections Between Items of Different Sections}
% 	\end{center}
% \end{figure}


% \begin{figure}[h!]
% 	\begin{center}
% 		%\rotatebox{-90}
% 		{
% 			\includegraphics[width=0.7\textwidth]{RTrace.png}
% 		}
% 		\caption{\label{Fig_RTrace} Traceability Matrix Showing the Connections Between Requirements, Instance Models, and Data Constraints}
% 	\end{center}
% \end{figure}

\newpage

\bibliographystyle {plainnat}
\bibliography {SRS}

\newpage

\section{Appendix}

\wss{Your report may require an appendix.  For instance, this is a good point to
show the values of the symbolic parameters introduced in the report.}

\subsection{Symbolic Parameters}

\wss{The definition of the requirements will likely call for SYMBOLIC\_CONSTANTS.
Their values are defined in this section for easy maintenance.}

\end{document}