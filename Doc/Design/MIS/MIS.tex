\documentclass[12pt, titlepage]{article}

\usepackage{amsmath, mathtools}

\usepackage[round]{natbib}
\usepackage{amsfonts}
\usepackage{amssymb}
\usepackage{graphicx}
\usepackage{colortbl}
\usepackage{xr}
\usepackage{hyperref}
\usepackage{longtable}
\usepackage{xfrac}
\usepackage{tabularx}
\usepackage{float}
\usepackage{siunitx}
\usepackage{booktabs}
\usepackage{multirow}
\usepackage[section]{placeins}
\usepackage{caption}
\usepackage{fullpage}

\hypersetup{
bookmarks=true,     % show bookmarks bar?
colorlinks=true,       % false: boxed links; true: colored links
linkcolor=red,          % color of internal links (change box color with linkbordercolor)
citecolor=blue,      % color of links to bibliography
filecolor=magenta,  % color of file links
urlcolor=cyan          % color of external links
}

\usepackage{array}

\input{../../Comments}

\newcommand{\progname}{LODES}
\newcommand{\progdesc}{Library of ODE Solvers}
\newcommand{\itab}[1]{\hspace{0em}\rlap{#1}}
\newcommand{\tab}[1]{\hspace{.05\textwidth}\rlap{#1}}
\newcommand{\iitab}[1]{\hspace{.3\textwidth}\rlap{#1}}

\begin{document}

\title{Module Interface Specification for \progname{}\\ (\progdesc{})}

\author{Paul Aoanan}

\date{\today}

\maketitle

\pagenumbering{roman}

\section{Revision History}

\begin{tabularx}{\textwidth}{p{3cm}p{2cm}X}
\toprule {\bf Date} & {\bf Version} & {\bf Notes}\\
\midrule
December 4, 2017 & 1.0 & Initial draft.\\
\today{} & 1.1 & Addressed comments and initial release.\\
%Date 2 & 1.1 & Notes\\
\bottomrule
\end{tabularx}

~\newpage

\section{Symbols, Abbreviations and Acronyms}

See SRS Documentation at the following Github link:\\
\url{https://github.com/aoananp/cas741/blob/master/Doc/SRS/CA.pdf}
%\wss{give url}

%\wss{Also add any additional symbols, abbreviations or acronyms}

\wss{You don't really use the same symbols as your SRS.  Instead, you seem to
  prefer symbols that are closer to the code.  I think you might find the MIS
  easier to read if you used the SRS notation.  For instance, $x_0$ is easier to
  read than x\_0.  If you want to use the code style symbols, you should show
  the mapping between the SRS symbols and the MIS symbols.}

\newpage

\tableofcontents

\listoftables

\newpage

\pagenumbering{arabic}

\section{Introduction}

The following document details the Module Interface Specifications for
\progname{}, the \progdesc{}.
%\wss{Fill in your project name and description}

Complementary documents include the System Requirement Specifications
and Module Guide.  The full documentation and implementation can be
found at the following link: \url{https://github.com/aoananp/cas741}. 


\section{Notation}

%\wss{You should describe your notation.  You can use what is below as a starting point.}

The structure of the MIS for modules comes from \citet{HoffmanAndStrooper1995},
with the addition that template modules have been adapted from
\cite{GhezziEtAl2003}.  The mathematical notation comes from Chapter 3 of
\citet{HoffmanAndStrooper1995}.  For instance, the symbol := is used for a
multiple assignment statement and conditional rules follow the form $(c_1
\Rightarrow r_1 | c_2 \Rightarrow r_2 | ... | c_n \Rightarrow r_n )$.

The following table summarizes the primitive data types used by \progname. 

\begin{center}
\renewcommand{\arraystretch}{1.2}
\noindent 
\begin{tabular}{l l p{7.5cm}} 
\toprule 
\textbf{Data Type} & \textbf{Notation} & \textbf{Description}\\ 
\midrule
character & char & a single symbol or digit\\
integer & $\mathbb{Z}$ & a number without a fractional component in (-$\infty$, $\infty$) \\
natural number & $\mathbb{N}$ & a number without a fractional component in [1, $\infty$) \\
real & $\mathbb{R}$ & any number in (-$\infty$, $\infty$)\\
\bottomrule
\end{tabular} 
\end{center}

\noindent
The specification of \progname \ uses some derived data types: sequences, strings, and
tuples. Sequences are lists filled with elements of the same data type. Strings
are sequences of characters. Tuples contain a list of values, potentially of
different types. In addition, \progname \ uses functions, which
are defined by the data types of their inputs and outputs. Local functions are
described by giving their type signature followed by their specification.

\section{Module Decomposition}

The following table is taken directly from the Module Guide document for this project.

\begin{table}[h!]
\centering
\begin{tabular}{p{0.3\textwidth} p{0.6\textwidth}}
\toprule
\textbf{Level 1} & \textbf{Level 2}\\
\midrule

{Hardware-Hiding Module} & ~ \\
\midrule

\multirow{5}{0.3\textwidth}{Behaviour-Hiding Module} & External Interface Module\\
& Euler's Method Module\\
& Trapezoidal Method Module\\
& Heun's Method Module\\
& Runge-Kutta's Method Module\\
\midrule

\multirow{2}{0.3\textwidth}{Software Decision Module} & {Equation String Parser Module}\\
%& Plotting Module\\
& Output Format Module\\
\bottomrule

\end{tabular}
\caption{Module Hierarchy}
\label{TblMH}
\end{table}

\newpage
~\newpage

\section{MIS of External Interface Module} \label{modExternalInterface}
This module is the interface exposed to the external world or driver program.
It provides access to the library and returns the solution to the ODE IVP.

\subsection{Module}
lodes

%\wss{Short name for the module}

\subsection{Uses}

EqParse (Section \ref{modEqParse}), Euler (Section \ref{modEuler}), Trap (Section \ref{modTrap}),
Heun (Section \ref{modHeun}), RK (Section \ref{modRK}), Output (Section \ref{modOutput})

\subsection{Syntax}

\begin{center}
\begin{tabular}{p{4cm} p{4cm} p{4cm} p{2cm}}
\hline
\textbf{Name} & \textbf{In} & \textbf{Out} & \textbf{Exceptions} \\
\hline
ODE\_method & ODE\_method $\in \{1, 2, 3, 4\}$  & - &  inputerror\\
ODE\_eq & string & - & badODEEq\\
x\_0 & $\mathbb{R}$ & - & badX0\\
y\_0 & $\mathbb{R}$ & - & badY0\\
x\_k & $\mathbb{R}$ & - & badXK\\
h & h such that h $\in \mathbb{R}$ and h $> 0$ & - & badH\\
plot & bool & - & - \\
displayResult & bool & - & - \\
x & - & [1 x $n$] $\in \mathbb{R}$ & - \\
y & - & [1 x $n$] $\in \mathbb{R}$ & - \\
success & - & BOOL & - \\
\hline
\end{tabular}
\end{center}

\wss{Rather than ODE methods being selected by an integer from 1 to 4, why not
  define a new enumerated type with the possible values Euler, Trap, etc.?}
  \wpa{Comment taken into account, but the design intent was to make it low-level in a sense
  that string manipulation shall be minimized.}

\subsection{Semantics}

\subsubsection{State Variables}
none

\newpage

\subsubsection{Access Routine Semantics}

\noindent lodes(ODE\_method, ODE\_eq, x\_0, y\_0, x\_k, h, plot, displayResult):
\wss{displayResult is missing here} \wpa{Thank you - added the variable.}

\begin{itemize}

\item Pseudocode:\\
%Get (ODE\_method: integer), (ODE\_eq: string), (x\_0: $\mathbb{R}$), (y\_0: $\mathbb{R}$),
%(x\_k: $\mathbb{R}$),\\
%(h: $\mathbb{R}$ . h $>$ 0) from input arguments\\
function f, bool eq\_OK $:=$ EqParse(ODE\_eq);\\
if NOT(eq\_OK)\\
\itab{ } \tab{return badODEEq := true, success := false;}\\
if NOT(ISREAL(x\_0))\\
\itab{ } \tab{return badX0 := true, success := false;}\\
if NOT(ISREAL(y\_0))\\
\itab{ } \tab{return badY0 := true, success := false;}\\
if NOT(ISREAL(x\_k))\\
\itab{ } \tab{return badXK := true, success := false;}\\
if NOT(ISREAL(h) and h $>$ 0)\\
\itab{ } \tab{return badH := true, success := false;}\\
Select ODE\_method:\\
\itab{ } \tab{Case: 1}\\
\itab{ } \tab{} \tab{x, y, success $:=$ euler(f, x\_0, y\_0, x\_k, h);}\\
\itab{ } \tab{Case: 2}\\
\itab{ } \tab{} \tab{x, y, success $:=$ trap(f, x\_0, y\_0, x\_k, h);}\\
\itab{ } \tab{Case: 3}\\
\itab{ } \tab{} \tab{x, y, success $:=$ heun(f x\_0, y\_0, x\_k, h);}\\
\itab{ } \tab{Case: 4}\\
\itab{ } \tab{} \tab{x, y, success $:=$ rk(f, x\_0, y\_0, x\_k, h);}\\
\itab{ } \tab{Case: else}\\
\itab{ } \tab{} \tab{return inputerror := true, success := false;}\\
End Select\\
\\
output(x, y, plot, displayResult);\\
\\
return x, y, success;
\end{itemize}

\newpage

\section{MIS of the Equation String Parser} \label{modEqParse}
This module handles the implementation of the Equation String Parser module. 

\subsection{Module}
eqParse

\subsection{Uses}

none

\subsection{eqParse}

\begin{center}
\begin{tabular}{p{4cm} p{4cm} p{4cm} p{2cm}}
\hline
\textbf{Name} & \textbf{In} & \textbf{Out} & \textbf{Exceptions} \\
\hline
ODE\_eq & string & - & - \\
f & - & Machine-interpreted equation & - \\
eq\_OK & - & bool & - \\
\hline
\end{tabular}
\end{center}

\wss{You misinterpreted the MIS syntax section.  The name is the name of the
  access program and the in and out are the types of the inputs and outputs.
  The name for your syntax section should be eqParse.}
  \wpa{Thank you - modifications done in the document.}

\subsection{Semantics}

\subsubsection{State Variables}
none

\subsubsection{Access Routine Semantics}
eqParse(ODE\_eq):
\begin{itemize}
\item Pseudocode:\\
try\\
\itab{ } \tab{f := parse(ODE\_eq); \text{\%convert the ODE equation string to a machine-interpretable string}}\\
\itab{ } \tab{return f, eq\_OK := true;}\\
catch\\
\itab{ } \tab{return f := 0, eq\_OK := false;}
\end{itemize}

\newpage


\section{MIS of Euler's Method} \label{modEuler}
This module handles the implementation of solving an ODE IVP using Euler's Method. 

\subsection{Module}
euler

\subsection{Uses}

None applicable.

\subsection{euler}

\begin{center}
\begin{tabular}{p{4cm} p{4cm} p{4cm} p{2cm}}
\hline
\textbf{Name} & \textbf{In} & \textbf{Out} & \textbf{Exceptions} \\
\hline
f & Machine-interpreted equation & - & -\\
x\_0 & $\mathbb{R}$ & - & -\\
y\_0 & $\mathbb{R}$ & - & -\\
x\_k & $\mathbb{R}$ & - & -\\
h & h such that h $\in \mathbb{R}$ and h $> 0$ & - & -\\
x & - & $\mathbb{R}^n$ & - \\
y & - & $\mathbb{R}^n$ & - \\
success & - & BOOL & - \\
\hline
\end{tabular}
\end{center}

\wss{The notation for a sequence of reals is odd.  It would be easier to
  understand if you just used $\mathbb{R}^n$ for the sequence of real numbers.}
  \wpa {Thank you - it has been modified.}

\wss{If your constraint on $h$ is violated, what happens?  I would think that
  there would be an exception.}
  \wpa{This constraint has been discussed in the Interface Module. No error handling
  is done in this module.}

\subsection{Semantics}

\subsubsection{State Variables}
none

\newpage

\subsubsection{Access Routine Semantics}
euler(f, x\_0, y\_0, x\_k, h):
\begin{itemize}
\item Pseudocode:\\
success := false;\\
x(1) := x\_0;\\
y(1) := y\_0;\\
N := (x\_0 - x\_k) / h;\\
for n = 1 to N\\
\itab{ } \tab{x(n+1) := x(n) + h;}\\
\itab{ } \tab{y(n+1) := y(n) + h * f(x(n), y(n));}\\
end for\\
success := true;\\
return x, y, success;
\end{itemize}

\newpage

\section{MIS of Trapezoidal Method} \label{modTrap}
This module handles the implementation of solving an ODE IVP using the Trapezoidal Method. 

\subsection{Module}
trap

\subsection{Uses}

None applicable.

\subsection{trap}

\begin{center}
\begin{tabular}{p{4cm} p{4cm} p{4cm} p{2cm}}
\hline
\textbf{Name} & \textbf{In} & \textbf{Out} & \textbf{Exceptions} \\
\hline
f & Machine-interpreted equation & - & -\\
x\_0 & $\mathbb{R}$ & - & -\\
y\_0 & $\mathbb{R}$ & - & -\\
x\_k & $\mathbb{R}$ & - & -\\
h & h such that h $\in \mathbb{R}$ and h $> 0$ & - & -\\
x & - & $\mathbb{R}^n$ & - \\
y & - & $\mathbb{R}^n$ & - \\
success & - & BOOL & - \\
\hline
\end{tabular}
\end{center}

\subsection{Semantics}

\subsubsection{State Variables}
none

\newpage

\subsubsection{Access Routine Semantics}
euler(f, x\_0, y\_0, x\_k, h):
\begin{itemize}
\item Pseudocode:\\
success := false;\\
double yNext = 0.0;\\
x(1) := x\_0;\\
y(1) := y\_0;\\
N := (x\_0 - x\_k) / h;\\
for n = 1 to N\\
\itab{ } \tab{x(n+1) := x(n) + h;}\\
\itab{ } \tab{\text{\%Solve for yNext, then store in y(n+1)}}\\
\itab{ } \tab{y(n+1) := solve(yNext := y(n) + (h/2) * f(x(n+1), yNext), yNext)};\\
end for\\
success := true;\\
return x, y, success;
\end{itemize}

\newpage

\section{MIS of Heun Method} \label{modHeun}
This module handles the implementation of solving an ODE IVP using Heun's Method. 

\subsection{Module}
heun

\subsection{Uses}

None applicable.

\subsection{heun}

\begin{center}
\begin{tabular}{p{4cm} p{4cm} p{4cm} p{2cm}}
\hline
\textbf{Name} & \textbf{In} & \textbf{Out} & \textbf{Exceptions} \\
\hline
f & Machine-interpreted equation & - & -\\
x\_0 & $\mathbb{R}$ & - & -\\
y\_0 & $\mathbb{R}$ & - & -\\
x\_k & $\mathbb{R}$ & - & -\\
h & h such that h $\in \mathbb{R}$ and h $> 0$ & - & -\\
x & - & $\mathbb{R}^n$ & - \\
y & - & $\mathbb{R}^n$ & - \\
success & - & BOOL & - \\
\hline
\end{tabular}
\end{center}

\subsection{Semantics}

\subsubsection{State Variables}
none

\newpage

\subsubsection{Access Routine Semantics}
heun(f, x\_0, y\_0, x\_k, h):
\begin{itemize}
\item Pseudocode:\\
success := false;\\
x(1) := x\_0;\\
y(1) := y\_0;\\
N := (x\_0 - x\_k) / h;\\
for n = 1 to N\\
\itab{ } \tab{x(n+1) := x(n) + h;}\\
\itab{ } \tab{y(n+1) := y(n) + (h/2) * (f(x(n) + h, y(n) + h * f(x(n), y(n)));}\\
end for\\
success := true;\\
return x, y, success;
\end{itemize}

\newpage

\section{MIS of Runge-Kutta 4 Method} \label{modRK}
This module handles the implementation of solving an ODE IVP using the Runge-Kutta 4 Method. 

\subsection{Module}
rk

\subsection{Uses}

None applicable.

\subsection{rk}

\begin{center}
\begin{tabular}{p{4cm} p{4cm} p{4cm} p{2cm}}
\hline
\textbf{Name} & \textbf{In} & \textbf{Out} & \textbf{Exceptions} \\
\hline
f & Machine-interpreted equation & - & -\\
x\_0 & $\mathbb{R}$ & - & -\\
y\_0 & $\mathbb{R}$ & - & -\\
x\_k & $\mathbb{R}$ & - & -\\
h & h such that h $\in \mathbb{R}$ and h $> 0$ & - & -\\
x & - & $\mathbb{R}^n$ & - \\
y & - & $\mathbb{R}^n$ & - \\
success & - & BOOL & - \\
\hline
\end{tabular}
\end{center}

\subsection{Semantics}

\subsubsection{State Variables}
none

\newpage

\subsubsection{Access Routine Semantics}
rk(f, x\_0, y\_0, x\_k, h):
\begin{itemize}
\item Pseudocode:\\
success := false;\\
x(1) := x\_0;\\
y(1) := y\_0;\\
double k1, k2, k3, k4;\\
N := (x\_0 - x\_k) / h;\\
for n = 1 to N\\
\itab{ } \tab{x(n+1) := x(n) + h;}\\
\itab{ } \tab{k1 := f(x(n), y(n));}\\
\itab{ } \tab{k2 := f(x(n) + h/2, y(n) + h * (k1/2));}\\
\itab{ } \tab{k3 := f(x(n) + h/2, y(n) + h * (k2/2));}\\
\itab{ } \tab{k4 := f(x(n) + h, y(n) + h * k3);}\\
\itab{ } \tab{y(n+1) := y(n) + (h/6) * (k1 + 2*k2 + 2*k3 + k4);}\\
end for\\
success := true;\\
return x, y, success;
\end{itemize}

\newpage

\section{MIS of the Output Module} \label{modOutput}
This module handles the implementation of the Output module. 

\subsection{Module}
output

\subsection{Uses}

Hardware Hiding (Section \ref{modHWH})

\subsection{output}

\begin{center}
\begin{tabular}{p{4cm} p{4cm} p{4cm} p{2cm}}
\hline
\textbf{Name} & \textbf{In} & \textbf{Out} & \textbf{Exceptions} \\
\hline
plot & BOOL & - & - \\
displayResult & BOOL & - & - \\
x & [1 x $n$] & - & - \\
y & [1 x $n$] & - & - \\
\hline
\end{tabular}
\end{center}

\subsection{Semantics}

\subsubsection{State Variables}
none

\subsubsection{Access Routine Semantics}
output(x, y, plot, displayResult):
\begin{itemize}
\item Pseudocode:\\
if (plot)\\
\itab { } \tab{plot(x, y);}\\
if (displayResult)\\
\itab { } \tab{print(x);}\\
\itab { } \tab{print(y);}\\
end if

\end{itemize}

\newpage

\section{MIS of the Hardware Hiding Module} \label{modHWH}
This module handles the implementation of the Output module. 

\wss{Unless you are writing device drivers, you don't really need to specify the
  access programs for the hardware hiding module.  As it is, your access
  programs are unclear.}

\subsection{Module}
hardwarehiding

\subsection{Uses}

none

\subsection{hardwarehiding}

\begin{center}
\begin{tabular}{p{4cm} p{4cm} p{4cm} p{2cm}}
\hline
\textbf{Name} & \textbf{In} & \textbf{Out} & \textbf{Exceptions} \\
\hline
x & [1 x $n$] & - & - \\
y & [1 x $n$] & - & - \\
\hline
\end{tabular}
\end{center}

\subsection{Semantics}

\subsubsection{State Variables}
none

\subsubsection{Access Routine Semantics}
plot(x, y):
\begin{itemize}
\item Pseudocode:\\
displayPlotToScreen(x, y); \text{\%display x vs. y graph on screen}
\end{itemize}
display(x):
\begin{itemize}
\item Pseudocode:\\
outputToScreen(x); \text{\%display array on screen}
\end{itemize}

\wss{I would have expected environment variables here to show the connection
  between your code and the file system and the screen.  I would expect the
  output related environment variables to be in your output module.  In this way
  you can hide the details of the hardware hiding module.}
  
 \wpa{The Hardware Hiding Module uses the same variables
 as the Output Module. The intent of this module is to bridge the gap
 between the OS and HW implementation with LODES.
 I am not familiar with the environment variables for MATLAB and MacOS (i.e. pixels, registry addresses, etc.)
 I will look into it further in the next revision of the MIS.}

\newpage

%\section{MIS of \wss{Module Name}} \label{Module} \wss{Use labels for cross-referencing}
%
%\subsection{Module}
%
%\wss{Short name for the module}
%
%\subsection{Uses}
%
%
%\subsection{Syntax}
%
%\subsubsection{Exported Access Programs}
%
%\begin{center}
%\begin{tabular}{p{2cm} p{4cm} p{4cm} p{2cm}}
%\hline
%\textbf{Name} & \textbf{In} & \textbf{Out} & \textbf{Exceptions} \\
%\hline
%\wss{accessProg} & - & - & - \\
%\hline
%\end{tabular}
%\end{center}
%
%\subsection{Semantics}
%
%\subsubsection{State Variables}
%
%
%\subsubsection{Access Routine Semantics}
%
%\noindent \wss{accessProg}():
%\begin{itemize}
%\item transition: \wss{if appropriate} 
%\item output: \wss{if appropriate} 
%\item exception: \wss{if appropriate} 
%\end{itemize}

\newpage

\section{MG to MIS Traceability Matrix}
\begin{table}[H]
\centering
\begin{tabular}{p{0.4\textwidth} p{0.5\textwidth}}
\toprule
\textbf{MG Module} & \textbf{MIS Module}\\
\midrule
M1 - Hardware Hiding & Section \ref{modHWH} - Hardware Hiding\\
M2 - External Interface & Section \ref{modExternalInterface} - External Interface\\
M3 - Equation String Parser & Section \ref{modEqParse} - Equation String Parser\\
M4 - Output Format Module & Section \ref{modOutput} - Output\\
M5 - Euler's Method & Section \ref{modEuler} - Euler's Method\\
M6 - Trapezoidal Method & Section \ref{modTrap} - Trapezoidal Method\\
M7 - Heun's Method & Section \ref{modHeun} - Heun's Method\\
M8 - Runge-Kutta 4 Method & Section \ref{modRK} - Runge-Kutta 4 Method\\

\bottomrule
\end{tabular}
\caption{Trace Between Module Guide and Module Interface Specification}
\label{TblRT}
\end{table}

\wss{Nice to show this.}

\newpage

\bibliographystyle {plainnat}
\bibliography {../../../ReferenceMaterial/References}

\newpage

\section{Appendix} \label{Appendix}

Not applicable. \wss{If it isn't applicable, you can comment out this section.}

\end{document}
