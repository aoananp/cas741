\documentclass[12pt, titlepage]{article}

\usepackage{amsmath, mathtools}

\usepackage[round]{natbib}
\usepackage{amsfonts}
\usepackage{amssymb}
\usepackage{graphicx}
\usepackage{colortbl}
\usepackage{xr}
\usepackage{hyperref}
\usepackage{longtable}
\usepackage{xfrac}
\usepackage{tabularx}
\usepackage{float}
\usepackage{siunitx}
\usepackage{booktabs}
\usepackage{multirow}
\usepackage[section]{placeins}
\usepackage{caption}
\usepackage{fullpage}

\hypersetup{
bookmarks=true,     % show bookmarks bar?
colorlinks=true,       % false: boxed links; true: colored links
linkcolor=red,          % color of internal links (change box color with linkbordercolor)
citecolor=blue,      % color of links to bibliography
filecolor=magenta,  % color of file links
urlcolor=cyan          % color of external links
}

\usepackage{array}

%% Comments

\usepackage{color}

\newif\ifcomments\commentstrue

\ifcomments
\newcommand{\authornote}[3]{\textcolor{#1}{[#3 ---#2]}}
\newcommand{\todo}[1]{\textcolor{red}{[TODO: #1]}}
\else
\newcommand{\authornote}[3]{}
\newcommand{\todo}[1]{}
\fi

\newcommand{\wss}[1]{\authornote{blue}{SS}{#1}}
\newcommand{\wpa}[1]{\authornote{magenta}{PA}{#1}}


\newcommand{\progname}{LODES}
\newcommand{\progdesc}{Library of ODE Solvers}
\newcommand{\itab}[1]{\hspace{0em}\rlap{#1}}
\newcommand{\tab}[1]{\hspace{.05\textwidth}\rlap{#1}}
\newcommand{\iitab}[1]{\hspace{.3\textwidth}\rlap{#1}}

\begin{document}

\title{Module Interface Specification for \progname{}\\ (\progdesc{})}

\author{Paul Aoanan}

\date{\today}

\maketitle

\pagenumbering{roman}

\section{Revision History}

\begin{tabularx}{\textwidth}{p{3cm}p{2cm}X}
\toprule {\bf Date} & {\bf Version} & {\bf Notes}\\
\midrule
\today{} & 1.0 & Initial draft.\\
%Date 2 & 1.1 & Notes\\
\bottomrule
\end{tabularx}

~\newpage

\section{Symbols, Abbreviations and Acronyms}

See SRS Documentation at the following Github link:\\
\url{https://github.com/aoananp/cas741/blob/master/Doc/SRS/CA.pdf}
%\wss{give url}

%\wss{Also add any additional symbols, abbreviations or acronyms}

\newpage

\tableofcontents

\newpage

\pagenumbering{arabic}

\section{Introduction}

The following document details the Module Interface Specifications for
\progname{}, the \progdesc{}.
%\wss{Fill in your project name and description}

Complementary documents include the System Requirement Specifications
and Module Guide.  The full documentation and implementation can be
found at the following link: \url{https://github.com/aoananp/cas741}. 
%\wss{provide the url for your repo}

\section{Notation}

%\wss{You should describe your notation.  You can use what is below as a starting point.}

The structure of the MIS for modules comes from \citet{HoffmanAndStrooper1995},
with the addition that template modules have been adapted from
\cite{GhezziEtAl2003}.  The mathematical notation comes from Chapter 3 of
\citet{HoffmanAndStrooper1995}.  For instance, the symbol := is used for a
multiple assignment statement and conditional rules follow the form $(c_1
\Rightarrow r_1 | c_2 \Rightarrow r_2 | ... | c_n \Rightarrow r_n )$.

The following table summarizes the primitive data types used by \progname. 

\begin{center}
\renewcommand{\arraystretch}{1.2}
\noindent 
\begin{tabular}{l l p{7.5cm}} 
\toprule 
\textbf{Data Type} & \textbf{Notation} & \textbf{Description}\\ 
\midrule
character & char & a single symbol or digit\\
integer & $\mathbb{Z}$ & a number without a fractional component in (-$\infty$, $\infty$) \\
natural number & $\mathbb{N}$ & a number without a fractional component in [1, $\infty$) \\
real & $\mathbb{R}$ & any number in (-$\infty$, $\infty$)\\
\bottomrule
\end{tabular} 
\end{center}

\noindent
The specification of \progname \ uses some derived data types: sequences, strings, and
tuples. Sequences are lists filled with elements of the same data type. Strings
are sequences of characters. Tuples contain a list of values, potentially of
different types. In addition, \progname \ uses functions, which
are defined by the data types of their inputs and outputs. Local functions are
described by giving their type signature followed by their specification.

\section{Module Decomposition}

The following table is taken directly from the Module Guide document for this project.

\begin{table}[h!]
\centering
\begin{tabular}{p{0.3\textwidth} p{0.6\textwidth}}
\toprule
\textbf{Level 1} & \textbf{Level 2}\\
\midrule

{Hardware-Hiding Module} & ~ \\
\midrule

\multirow{5}{0.3\textwidth}{Behaviour-Hiding Module} & External Interface Module\\
& Euler's Method Module\\
& Trapezoidal Method Module\\
& Heun's Method Module\\
& Runge-Kutta's Method Module\\
\midrule

\multirow{2}{0.3\textwidth}{Software Decision Module} & {Equation String Parser Module}\\
%& Plotting Module\\
& Output Format Module\\
\bottomrule

\end{tabular}
\caption{Module Hierarchy}
\label{TblMH}
\end{table}

\newpage
~\newpage

\section{MIS of External Interface Module} \label{modExternalInterface}
This module is the interface exposed to the external world or driver program.
It provides access to the library and returns the solution to the ODE IVP.

\subsection{Module}
lodes

%\wss{Short name for the module}

\subsection{Uses}

EqParse, Euler (Section \ref{modEuler}), Trap (Section \ref{modTrap}),
Heun (Section \ref{modHeun}), RK (Section \ref{modRK}), Output (Section \ref{modOutput})

\subsection{Syntax}

\begin{center}
\begin{tabular}{p{4cm} p{4cm} p{4cm} p{2cm}}
\hline
\textbf{Name} & \textbf{In} & \textbf{Out} & \textbf{Exceptions} \\
\hline
ODE\_method & ODE\_method $\in \{1, 2, 3, 4\}$  & - &  inputerror\\
ODE\_eq & string & - & badODEEq\\
x\_0 & $\mathbb{R}$ & - & badX0\\
y\_0 & $\mathbb{R}$ & - & badY0\\
x\_k & $\mathbb{R}$ & - & badXK\\
h & h such that h $\in \mathbb{R}$ and h $> 0$ & - & badH\\
y & - & [$n$ x $1] \in \mathbb{R}$ & - \\
success & - & BOOL & - \\
\hline
\end{tabular}
\end{center}

\subsection{Semantics}

\subsubsection{State Variables}
none

\subsubsection{Access Routine Semantics}

\noindent lodes():
\begin{itemize}

\item Pseudocode:\\
Get (ODE\_method: integer), (ODE\_eq: string), (x\_0: $\mathbb{R}$), (y\_0: $\mathbb{R}$),
(x\_k: $\mathbb{R}$),\\
(h: $\mathbb{R}$ . h $>$ 0) from input arguments\\
\\
bool eq\_OK $:=$ EqParse(ODE\_eq)\\
\\
if NOT(eq\_OK)\\
\itab{ } \tab{return success = false}\\
\\
Select ODE\_method:\\
\itab{ } \tab{Case: 1}\\
\itab{ } \tab{} \tab{y, success $:=$ euler(ODE\_eq, x\_0, y\_0, h)}\\
\itab{ } \tab{Case: 2}\\
\itab{ } \tab{} \tab{y, success $:=$ trap(ODE\_eq, x\_0, y\_0, h)}\\
\itab{ } \tab{Case: 3}\\
\itab{ } \tab{} \tab{y, success $:=$ heun(ODE\_eq, x\_0, y\_0, h)}\\
\itab{ } \tab{Case: 4}\\
\itab{ } \tab{} \tab{y, success $:=$ rk(ODE\_eq, x\_0, y\_0, h)}\\



\end{itemize}

\newpage


\section{MIS of Euler's Method} \label{modEuler}
This module handles the implementation of solving an ODE IVP using Euler's Method. 

\subsection{Module}
euler

%\wss{Short name for the module}

\subsection{Uses}

None applicable.

\subsection{Syntax}

\begin{center}
\begin{tabular}{p{4cm} p{4cm} p{4cm} p{2cm}}
\hline
\textbf{Name} & \textbf{In} & \textbf{Out} & \textbf{Exceptions} \\
\hline
ODE\_method & ODE\_method $\in \{1, 2, 3, 4\}$  & - &  inputerror\\
ODE\_eq & STRING & - & badODEEq\\
x\_0 & $\mathbb{R}$ & - & badX0\\
y\_0 & $\mathbb{R}$ & - & badY0\\
x\_k & $\mathbb{R}$ & - & badXK\\
h & h such that h $\in \mathbb{R}$ and h $> 0$ & - & badH\\
y\_k & - & $\mathbb{R}$ & - \\
success & - & BOOL & - \\
\hline
\end{tabular}
\end{center}

\subsection{Semantics}

\subsubsection{State Variables}
none

\subsubsection{Access Routine Semantics}

\noindent \wss{accessProg}():
\begin{itemize}
\item transition: \wss{if appropriate} 
\item output: \wss{if appropriate} 
\item exception: \wss{if appropriate} 
\end{itemize}

\newpage


\section{MIS of \wss{Module Name}} \label{Module} \wss{Use labels for cross-referencing}

\subsection{Module}

\wss{Short name for the module}

\subsection{Uses}


\subsection{Syntax}

\subsubsection{Exported Access Programs}

\begin{center}
\begin{tabular}{p{2cm} p{4cm} p{4cm} p{2cm}}
\hline
\textbf{Name} & \textbf{In} & \textbf{Out} & \textbf{Exceptions} \\
\hline
\wss{accessProg} & - & - & - \\
\hline
\end{tabular}
\end{center}

\subsection{Semantics}

\subsubsection{State Variables}


\subsubsection{Access Routine Semantics}

\noindent \wss{accessProg}():
\begin{itemize}
\item transition: \wss{if appropriate} 
\item output: \wss{if appropriate} 
\item exception: \wss{if appropriate} 
\end{itemize}

\newpage

\bibliographystyle {plainnat}
\bibliography {../../../ReferenceMaterial/References}

\newpage

\section{Appendix} \label{Appendix}

\wss{Extra information if required}

\end{document}
