\documentclass[12pt, titlepage]{article}

\usepackage{booktabs}

\usepackage{amsmath, mathtools}
\usepackage{amsfonts}
\usepackage{amssymb}
\usepackage{graphicx}
\usepackage{colortbl}
\usepackage{xr}
\usepackage{hyperref}
\usepackage{longtable}
\usepackage{xfrac}
\usepackage{tabularx}
\usepackage{float}
\usepackage{siunitx}
\usepackage{booktabs}
\usepackage{caption}
\usepackage{pdflscape}
\usepackage{afterpage}

\usepackage[round]{natbib}
\hypersetup{
    colorlinks,
    citecolor=black,
    filecolor=black,
    linkcolor=red,
    urlcolor=blue
}
\usepackage[round]{natbib}

%% Comments

\usepackage{color}

\newif\ifcomments\commentstrue

\ifcomments
\newcommand{\authornote}[3]{\textcolor{#1}{[#3 ---#2]}}
\newcommand{\todo}[1]{\textcolor{red}{[TODO: #1]}}
\else
\newcommand{\authornote}[3]{}
\newcommand{\todo}[1]{}
\fi

\newcommand{\wss}[1]{\authornote{blue}{SS}{#1}}
\newcommand{\wpa}[1]{\authornote{magenta}{PA}{#1}}


\newcommand{\famname}{LODES} % PUT YOUR PROGRAM NAME HERE
\newcommand{\famdesc}{Library of ODE Solvers}
\newcommand{\famurl}{https://github.com/aoananp/cas741/}

\newcounter{tnum} %Test Number
\newcommand{\ttheccnum}{T\thetnum}
\newcommand{\tref}[1]{T\ref{#1}}

\usepackage{fullpage}

\begin{document}

\title{Test Report: \famdesc{} (\famname{})} 
\author{Paul Aoanan}
\date{\today}
	
\maketitle

\pagenumbering{roman}

\section{Revision History}

\begin{tabularx}{\textwidth}{p{3cm}p{2cm}X}
\toprule {\bf Date} & {\bf Version} & {\bf Notes}\\
\midrule
\today & 1.0 & Initial draft.\\
%Date 2 & 1.1 & Notes\\
\bottomrule
\end{tabularx}

~\newpage

\section{Symbols, Abbreviations and Acronyms}

\subsection{Table of Symbols}

The table that follows summarizes the symbols used in this document along with
their units.  The choice of symbols was made to be consistent with the numeral analysis
and ordinary differential equation literature and with existing documentation
for solving ordinary differential equations.  The symbols are listed in alphabetical order.

\renewcommand{\arraystretch}{1.2}
\noindent \begin{longtable*}{l l l p{8cm}} \toprule
\textbf{symbol} & \textbf{unit} & \textbf{type} & \textbf{description}\\
\midrule
$dy/dx$ & \text {-} & $\mathbb{R}$ & Rate of change of $y$ with respect to $x$\\
$\varepsilon_\text{relative}$ & - & $\mathbb{R}$ & The relative error\\
$F$ & \text{-} & $\mathbb{R}^3 \rightarrow \mathbb{R}$ & Order function applied to the Runge Kutta Method\\
$f(x, y)$ & \text{-} & $\mathbb{R}^2 \rightarrow \mathbb{R}$& Explicit form of the ODE function containing $(x,y)$\\
$f_{x}(x, y)$ & \text{-} & $\mathbb{R}^2 \rightarrow \mathbb{R}$ & Explicit form of the derivative of $f(x, y)$ with respect to x\\
$f_{y}(x, y)$ & \text{-} & $\mathbb{R}^2 \rightarrow \mathbb{R}$ & Explicit form of the derivative of $f(x, y)$ with respect to y\\
$h$ & \text{-} & $\mathbb{R}$ . $h > 0$ & Step-size from $x_\text{(0)}$ to the next point $x_\text{(1)}$, where $x_\text{(1)} = x_\text{(0)} + h$\\
$K_1, K_2, K_3, K_4$ & $\mathbb{R}$ & \text{-} & Intermediary variables used in the Runge-Kutta method\\
$n$ & \text{-} & $\mathbb{R}$ & Reference recursion step\\
$o$ & \text{-} & $\mathbb{R}$  & Solution variable\\
T & - & - & Test\\
$x_\text{0}$ & \text{-} & $\mathbb{R}$  & Initial value $x$\\
$x_\text{k}$ & \text{-} & $\mathbb{R}$  & Final value $x$\\
$x_\text{n}$ & \text{-} & $\mathbb{R}$ & Intermediate $n^\text{th}$ value $x$\\
$y_\text{0}$ & \text{-} & $\mathbb{R}$ & Initial value $y$\\ 
$y_\text{k}$ & \text{-} & $\mathbb{R}$  & Final value $y$\\
$y_\text{n}$ & \text{-} &$\mathbb{R}$ & Intermediate $n^\text{th}$ value $y$\\ 
$y'$ & \text{-} &$\mathbb{R} \rightarrow \mathbb{R}$& Implicit form of the first order ODE = $f(x, y)$\\
$y^\text{(n)}$ & \text{-} &$\mathbb{R} \rightarrow \mathbb{R}$& Implicit form of the ODE to the $n^\text{th}$ order\\
$y$ & \text{-} & $1$ x $\mathbb{R}^k$ & The array containing all intermediate $y_\text{n}$ values\\
\bottomrule
\end{longtable*}

\wss{symbols, abbreviations or acronyms -- you can reference the SRS tables if needed}

\newpage

\tableofcontents

\listoftables %if appropriate

\listoffigures %if appropriate

\newpage

\pagenumbering{arabic}

This document ..

\section{Functional Requirements Evaluation}

\section{Nonfunctional Requirements Evaluation}

\subsection{Usability}
		
\subsection{Performance}

\subsection{etc.}
	
\section{Comparison to Existing Implementation}	

This section will not be appropriate for every project.

\section{Unit Testing}

\section{Changes Due to Testing}

\section{Automated Testing}
		
\section{Trace to Requirements}
		
\section{Trace to Modules}		

\section{Code Coverage Metrics}

\bibliographystyle{plainnat}

\bibliography{SRS}

\end{document}
